\pdfbookmark[0]{English title page}{label:titlepage_en}
\aautitlepage{
  \englishprojectinfo{
    TBD
  }{
    NAN
  }{Spring Semester 2019}{SW611F19}{%
    Morten Hartvigsen (mhartv16)\\
    Mikkel Holm Jessen (mjesse12)\\
    Rikke Holm Jessen (rhja16)\\
    Bogi Napoleon Wennerstrøm (bwenne16)
  }{Ulrik Nyman\\
    Tiantian Liu}{1}{
    \today
  }
}{
  \textbf{Institute of Computer Science}\\
  Aalborg University\\
  \href{http://www.aau.dk}{http://www.aau.dk}
}{ 
As software developers, we need to be able to collaborate with other developers both in a team and in a larger work environment. In this project, we work on the GIRAF project as one of seven teams this year. We continue the development done by previous students on the GIRAF project and work primarily on the Weekplanner application. The application starts in the Xamarin framework, but we move it to the Flutter framework, programming in the dart language, and structure it with the BLoC pattern. The seven development teams work full stack across both frontend, server, and backend. The teams coordinate using a Scrum of Scrums process composed of four sprints. The process is revised every sprint and changed as needed. We describe the collaboration experience we got from working in both our group and coordinating with the other groups.
}

\cleardoublepage
