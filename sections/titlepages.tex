\pdfbookmark[0]{English title page}{label:titlepage_en}
\aautitlepage{
  \englishprojectinfo{
    TBD
  }{
    NAN
  }{Spring Semester 2019}{SW611F19}{%
    Morten Hartvigsen (mhartv16)\\
    Mikkel Holm Jessen (mjesse12)\\
    Rikke Holm Jessen (rhja16)\\
    Bogi Napoleon Wennerstrøm (bwenne16)
  }{Tiantian Liu}{1}{
    \today
  }
}{
  \textbf{Institute of Computer Science}\\
  Aalborg University\\
  \href{http://www.aau.dk}{http://www.aau.dk}
}{ 
As a software developer you need to be able to collaborate with other people both in a team and in a larger work environment. This project we work on the GIRAF project as one of seven teams this year. We continue work from previous students on the GIRAF project, and work primarily on the Weekplanner application. The application starts in the Xamarin framework, but we move it to the Flutter framework, programming in the dart language, and structure it with the BLoC pattern. The seven development teams work full stack and across both frontend, server and backend. The teams coordinate with a Scrum of Scrums process with four sprints. The process is revised every sprint, and changed as needed. We describe the collaboration experience we got with working in both our group and coordinating with the other groups.
}

\cleardoublepage
