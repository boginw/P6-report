%Term Example definitions
\newglossaryentry{utc}{name=UTC, description={Coordinated Universal Time}}
\newglossaryentry{adt}{name=ADT, description={Atlantic Daylight Time}}
\newglossaryentry{est}{name=EST, description={Eastern Standard Time}}
%\newglossaryentry{mvvm}{name=MVVM, description={Model View View-Model}}
\newglossaryentry{fullStack}{
    name={full Stack Team},
    description ={Team that completes full user stories, rather than only working in a specific part of the system.}
}
\newglossaryentry{skillGroup}{
    name={skill group},
    description = {A group formed by one person from each development team. The members of the groups have to know the current state of a specific part of the system}
}
\newglossaryentry{devTeam}{
    name={development team},
    description = {A small group of students developing small parts of the system together}
}
\newglossaryentry{team}{
    name={team},
    description = {Another name for a development team.}
}
\newglossaryentry{SOSSprintPlanning}{
    name={SOS Sprint Planning},
    description = {Planning the sprint for the whole GIRAF year.}
}
\newglossaryentry{SOSStandUp} {
    name={SOS Stand Up},
    description={Weekly standup meetings for the GIRAF year.}
}
\newglossaryentry{ReleasePreparation}{
    name={Release Preparation},
    description={Preparation for the sprint release at the end of each sprint. Involves all development teams}
}
\newglossaryentry{SOSSprintRetrospective}{
    name={SOS Sprint Retrospective},
    description={Sprint Retrospective for the whole development year.}
}
\newglossaryentry{SOSSprintReview}{
    name={SOS Sprint Review},
    description={Sprint Retrospective for the whole GIRAF development year.}
}

\newglossaryentry{Scrum_principles}{
    name={Scrum principles},
    description={A set og principles defining the Scrum process \cite{Sommerville:2015} }
}
\newglossaryentry{driver}{
    name={driver},
    description={The person that codes in TDD}
}
\newglossaryentry{api}{name=API, description={Application Programming Interface}}
\newglossaryentry{er}{name=ER, description={Entity-Relationship}}

\newglossaryentry{fapi}{name=api client, description={The part of the frontend that communicates with the API}}
\newglossaryentry{screen}{name=screen, description={}}

\newglossaryentry{ci}{name=CI, description={Continous Integration}}


%Acronym Example definitions

\newacronym{ui}{UI}{User interfase}
\newacronym{SMT}{SMT}{Scrum master team}
\newacronym{PO}{PO}{Product Owner}
\newacronym{Aadt}{ADT}{Atlantic Daylight Time}
\newacronym{Aest}{EST}{Eastern Standard Time}

\newacronym{Mvvm}{MVVM}{Model View View-Model}
\newacronym{Dto}{DTO}{Data Transfer Object}
\newacronym{G19}{G19}{GIRAF development year 2019}
\newacronym{G18}{G18}{GIRAF development of 2018}
\newacronym{XP}{XP}{Extreme programming}
\newacronym{T11}{T11}{Team 11 of the GIRAF development teams of 2019}
\newacronym{POT}{POT}{Project Owner team of 2019}
\newacronym{ST}{ST}{Scrum team of 2019}
\newacronym{SOS}{SOS}{Scrum of Scrums}
\newacronym{bloc}{BLoC}{Business Logic Component}

\newacronym{dto}{dto}{Data Transfer Object}
\newacronym{po}{po}{Project Owner}
\newacronym{xaml}{XAML}{Extensible Application Markup Lanugage}
% Standard command
%\gls{⟨label⟩}
% Capitalize first letter
%\Gls{⟨label⟩}
% Pluralize term
%\glspl{⟨label⟩}
%Capitalize and pluralize term
%\Glspl{⟨label⟩}
