\section{Cross teams process}
In the beginning of the semester, all groups had to define the process for \gls{G19}. 
This resulted in a manual\cite{processManual}, written by \gls{POT}, defining the practices and process to be followed by the groups. The decisions described in the following, are defined in the manual. 


For years, the GIRAF project has been run using a Scrum process. However, there have been issues with the normal Scrum approach other years, because every team worked on a different part of the system\cite{fullStackVSSpecific}. This meant that the \glspl{devTeam} needed a lot of communication between the groups. Because of this, the \gls{G19} process uses a technique called \gls{SOS}. This a technique for running a Scrum process in bigger teams and is adapted to fit the needs of \gls{G19}.
Here, the teams are organized as \Glspl{fullStack}. This means, that the teams work on full user stories rather than on a specific areas. 

\subsection*{Process activities}

To make sure all \glspl{devTeam} have enough knowledge in the specific areas of the system, there are dedicated \glspl{skillGroup}. There are a member from each \gls{devTeam} in every \gls{skillGroup}, and these members are expected to have knowledge about a specific area of the system. There are three different \glspl{skillGroup}, frontend, backend and server. 

The activities resembles the normal Scrum activities, but instead of \gls{devTeam} having the Scrum activities and artefacts by them selfs, each team can be seen as a member of the \gls{G19} team. Internally, each \gls{devTeam} has their own process. There are planned a total of four sprints, and in the following is described the activities in each sprint:

\begin{itemize}
    \item \Gls{SOSSprintPlanning}
    \item \Gls{SOSStandUp}
    \item \Gls{skillGroup} Meetings
    \item \Gls{ReleasePreparation}
    \item \Gls{SOSSprintRetrospective}
    \item \Gls{SOSSprintReview}
    \item Release Party
\end{itemize}

The SOS process is graphically represented in \autoref{fig:SOSProcess}. Here, the circles represent an activity. The stickmen represents a group member with a role. The "any member" role represents all of the group members. If when one stickman is placed in an activity, it means that one person from the group has to attend. 
One of each \gls{skillGroup} charecters are placed in the \gls{skillGroup} meeting activity. That means that it is them that attends the \gls{skillGroup} meetings.
When no stickman is placed in an activity, that means, that the activity is for everybody. 
The activities are ordered chronologically. Activities placed on the loop are repeated.

\begin{figure}[h]
    \begin{center}
        \includegraphics[width=0.95\textwidth]{figures/SOSProcess}
    \end{center}
    \caption{Representation of the \gls{SOS} process.}
    \label{fig:SOSProcess}
\end{figure}

The seven groups are represented with a cloud. \gls{SOS} only describes the process between \glspl{devTeam}. It is up to each group to define their own process, as long as it fits within the \gls{SOS} guidelines. 

The following part will contain a description of each of the activities.

\Gls{SOSSprintPlanning} is a meeting held on the first day of each sprint. All members of \Gls{G19} attends. Before the \Gls{SOSSprintPlanning}, the \gls{POT} has prepared user stories for the sprint backlog, and the purpose of the sprint. 
This is then presented for all \glspl{devTeam}, and the teams will then choose which user stories to implement, and estimate the chosen stories. 
The members of the \gls{POT} is available for questions, and needs to accept the other groups chosen stories and estimates. 
When the \glspl{devTeam} has found a number of tasks appropriate for them, they should leave the meeting. 
 
\Gls{SOSStandUp} is a meeting that is held one to three times a week. The purpose of this meeting, is to share the progress and challenges between the teams, so the different teams knows how the system is progressing, and can help each other with questions. 
Each group should send as few as possible people to answer the questions. The meeting should not take more than 15 minutes. 

The \Gls{skillGroup} meetings, are held so the members of each \gls{skillGroup} is kept up to date with what is going on in his part of the system. The goal is to have at least one meeting every week. 

\Gls{ReleasePreparation} covers four days before the end of each sprint. Here, the system are made ready for release, and the new features and changes are validated and verified. 

\Gls{SOSSprintReview} is held in the end of a sprint. Here only one group member may attend. In this meeting, each group representative talkes about what that group has been working on during the sprint. The progress on all user stories is evaluated. 

In the \gls{SOSSprintRetrospective} the development process is evaluated. Here all group members participates. Beforehand, the groups have to reflect on how they feel the \gls{SOS} process has been. There are formed small group containing approximately one member from each group. The members discuss how they interpret their groups findings. The feedback is collected, and the ideas is voted on. 

The last activity, the Release Party, is held as the last thing in each Sprint. The purpose here is to make enhance the \glspl{devTeam} feeling of ownership of the GIRAF project, and to enhance communication between the \glspl{devTeam}. If the different groups want to, they can present the work they have done during the Sprint.

\subsection*{Practices and standards} 
In the manual, some standards and practices the \glspl{devTeam} has to follow. These fit into the following categories:
\begin{itemize}
    \item Coding standards
    \item Documentation standards
    \item Version control
    \item Code review
\end{itemize}

The coding standards involves naming conventions, descriptions and a few other aspects. This is done so the code is easier to read for people that need to get familiar with the system, mainly the following years \glspl{devTeam}, because we found it hard to get to know the system when it was made by a lot of different people, with a lot of different coding standards.
It was also decided, that the \glspl{devTeam} should document the implemented components, their responsibility and how it interacts with other components. A good documentation of this makes it easier for following years. 
Last year, the project and repository was moved to Gitlab that now takes care of the version control. In the beginning of this semester our team moved it to Github. This is documented in section !!Ref til sektion!!. The workflow of the project is done using GitFlow Workflow. In the manual is described the rules for using branches. 
The point of the code review is to make sure that all code committed by the \glspl{devTeam} follows the coding standards. When a Team try to commit code, two people from different groups have to review it. They have to go through a checklist, to see if everything is alright. 

If a member of a team finds an issue in the program, he can make an issue report. This makes it possible to document errors systematically, without the \glspl{team} having to fix it themselves. If a \gls{team} however finds it important to introduce a new task to the product backlog, the team can issue a task creation request. The \gls{POT} then decides if they find it relevant to have in the product backlog. 

The process between the \gls{G19} \glspl{devTeam} is defined in the manual. This process will be evaluated at each \gls{SOSSprintRetrospective}, and changed as needed. It is designed to make the collaboration between the \glspl{team} run as smoothly as possible, without interfering with the \glspl{team} internal processes.  