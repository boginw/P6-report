%What is Giraf?
%What is the point of the project?
\section{What is Giraf} % (fold)
\label{cha:What is Giraf}

% chapter Giraf semester project (end)
The following section briefly describes what the Giraf project is, and how it is developed. The information in the section is primarily based on information from the Giraf website\ref{GirafWebsite}.
\\
Giraf is a project that focuses on making a digital tablet environment to use as a tool for autistic people with limited verbal communication skills. The project are developed by software engineering students as as a part of their bachelor projects.  The software engineering students work together in multiple teams, and coordinates the work between themselves. The project is made in collaboration with Børnehaven Birken (Kindergarten)\cite{bhBirken}, Egebakken (School)\cite{egebakken}, Enterne (Home for disabled)\cite{enterne}, and the speech institute at Aalborg municipality og Center for Autism and ADHD\cite{center_for_autism}. The project has run since 2011, and each year, the students continue where the last year students left off. 

The purpose of the project is to teach software engineering students how to work together in multiple groups. Since the project has run over multiple years, with new student teams each year, problems arise as the students has to learn the state of the project, and has only four months to work on it, and then hands it over to the next years students. If the next year has different ideas, or does not get a full picture of the project, it can become unstructured or, like in the previously mentioned case where changing the backend resulting in most of the frontend apps not working. Furthermore, if there is no hard set coding or documentation standards, it can be hard to get a clear understanding of what is happening and why.

There are already developed many tools, among them both games and a week schedule. However, many of the apps that worked at one point, stopped working after changes in the backend architecture\cite{AppsStatus2019}. 