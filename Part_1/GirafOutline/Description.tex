%What is Giraf?
%What is the point of the project?
\section{What is Giraf} % (fold)
\label{cha:What is Giraf}

% chapter Giraf semester project (end)
The following section briefly describes what the Giraf project is and how it is developed. The information in the section is primarily based on information from the Giraf website\cite{GirafWebsite}.
\\
Giraf is a project that focuses on making a digital tablet environment to use as a tool for autistic people with limited verbal communication skills. The project is developed by software engineering students, as as a part of their bachelor projects.  The software engineering students work in small teams, and coordinates the work between the teams. The project is made in collaboration with Børnehaven Birken (Kindergarten)\cite{bhBirken}, Egebakken (School)\cite{egebakken}, Enterne (Home for disabled)\cite{enterne}, and the speech institute at Aalborg municipality og Center for Autism and ADHD\cite{center_for_autism}. The project has run since 2011, and each year, the students continue where the last year students left off. 

The purpose of the project is to teach software engineering students how to work together in multiple groups in larger project. 

At the beginning of the project, it was decided to, like previous years, run the project with the Scrum principles. One group was appointed \gls{PO} team, and another was appointed Scrum team. The \gls{PO} team had the responsibility of talking to the costumer, and making the product backlog. The Scrum team had the responsibility of deciding how the project should run, and witch guidelines the groups had to follow. 

\section{Current state of Giraf}

Over the years the Giraf project has run, many different apps has been developed. However, when many different groups, has worked on the same project over many years,  resulted in some problems. 
These problems comes from different ideas, coding styles and an incomplete mental model of the system, and has resulted in an incoherent structure of the program. 
This also had the effect, that only one app, named Weekplanner, works at this time. This is because the groups in one year changed the backend of the system, leaving all other apps unusable. 
However, this is the only app that has been worked on the last couple of years, and after talking to the costumer, the \gls{PO} group also decided to only work on the Weekplanner app.   

Over the years the Giraf project has run, many different apps has been developed. However, at this time, only one app works. This app is called the Weekplanner. This is a result of changes to the backend without consideration of the structure of the other apps\cite{AppsStatus2019}. 
The weekplanner, however, is the only app that has been focus on in the last years. After the \gls{PO} group talked to the costumer, it was decided to also focus on the Weekplanner this year. Thus, this is the only app that will be described. 
The weekplanner app is meant as a tool for showing autistic people what they will do for a week. Each day has some activities associated with it, each of them represented as a picture. The activities are then ordered the first being on top, and later activities going down. 

%Sæt et billede ind her.

The weekplanner app is currently running Xamarin, and is only able to run on android devices. The frontend of the app is build with the \gls{Mvvm} pattern. 
Backend is... %Jeg er ikke helt sikker på det her, vide mere for at skrive. 
