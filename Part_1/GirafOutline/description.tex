GIRAF is a tablet enviroment development to ease communication with autistic people with little or no verbal communication. The project is developed by Software students as a part of their bachelor project. The students work in small teams and coordinate the work between teams. The following section descibes the GIRAF project briefly.\\

The project is in collaboration with the following institutes\cite{GirafWebsite}, that acts as customers in the development process.\\

\begin{itemize}
    \item Børnehaven Birken (Kindergarten) \cite{bhBirken}
    \item Egebakken (School) \cite{egebakken}
    \item Enterne (Home for disabled) \cite{enterne}
    \item The speech institute at Aalborg municipality
    \item Center for Autism and ADHD \cite{center_for_autism}
\end{itemize}

The project started in 2011, and each year, the students continue where the last year students left off.

At the beginning of the project, it was decided to, like previous years, run the project with the \gls{Scrum_principles}. One group was appointed \gls{PO} team and another was appointed \gls{SMT}. The \gls{PO} team had the responsibility of talking to the customer and making the product backlog. The Scrum master team had the responsibility of deciding and facilitate how the collaboration in the project should run, and which guidelines the groups had to follow.\\