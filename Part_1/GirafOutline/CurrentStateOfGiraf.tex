\section{Current state of Giraf}

When we started the semester we were handed the work done by previous years. In this section we describe some of the elements of the material we got as they looked at the time of handof.//

The Giraf project has produced many different apps over the years. Most of the apps are not working, wich is a result of the many different team that have worked on the project for many years.//

These problems come from different ideas, coding styles, and an incomplete mental model of the system, and has resulted in an incoherent structure of the project. Because the groups in 2017 \cite{SW608F18} changed the backend of the system, it left all apps unusable except the Weekplanner, since this is the only app they had time to update to reflect the new backend.//

The Weekplanner app is the only app that has received updates in the last couple of years, and with this in mind, the \gls{PO} group also decided to limit our development to the Weekplanner app this year.//

The Weekplanner app is a tool for showing autistic people what their plan is for the week. Each day has activities associated with it, ordered in chronologically, each of them represented as a picture also called a pictogram. The state of the app from the start of 2019, can be seen in \autoref{fig:WeekPlannerPicture}.//

\begin{figure}[ht]
        \begin{center}
            \includegraphics[width=0.95\textwidth]{figures/WeekPlannerPicture}
        \end{center}
        \caption{Weekplanner state at the start of 2019}
        \label{fig:WeekPlannerPicture}
\end{figure}

The Weekplanner app is using the Xamarin framework and can run on both Android and IOS devices.//
The backend is a .NET Core 2 project that uses traditional MVC. The backend supports all the capabilities of the current Weekplanner app. It exposes a REST-inspired API to the front end.//