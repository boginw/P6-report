\chapter{GIRAF Outline}

GIRAF is a tablet environment developed to ease communication between \glspl{guardian} and \glspl{citizen} with little or no verbal communication. Software students develop the project as a part of their bachelor project. The students work in small teams and coordinate the work between teams. The project is in collaboration with the following institutes \cite{GirafWebsite}, that acts as customers in the development process.

\begin{itemize}
    \item Børnehaven Birken (Kindergarten) \cite{bhBirken}
    \item Egebakken (School) \cite{egebakken}
    \item Enterne (Home for disabled) \cite{enterne}
    \item The speech institute at Aalborg municipality
    \item Center for Autism and ADHD \cite{center_for_autism}
\end{itemize}

The project started in 2011, and each year, the students continue where the last year students left off.

This semester started with the decision to, like previous years, run the project with the \gls{Scrum_principles}. One group was appointed \gls{PO} team and another was appointed \gls{SMT}. The \gls{PO} team had the responsibility of talking to the customer and making the product backlog. The \gls{SMT} had the responsibility of deciding and facilitate how the collaboration in the project should run, and which guidelines the groups had to follow.

\section{Current state of GIRAF}

When we started the semester, the work done by previous teams was handed over to us. In this section, we describe some of the elements of the material we got as they looked at the time of handover.

The GIRAF project has produced many different apps over the years. Most of the apps are not working, because the \glspl{devTeam} in 2017 changed the backend. They only had time to update one application to reflect the new backend, and therefore only one is working. This application is called Weekplanner. This is described in section \ref{sec:TheWeekplannerApplication}.

Because the weekplanner application is the only application that have been updated in the last couple of years, and the costumers pointed this out as the most important one, we decided to limit our development to the Weekplanner app this year.

The Weekplanner app is using the Xamarin framework and can run on both Android and IOS devices.

The backend is a .NET Core 2 project that uses traditional MVC and supports all the capabilities of the current Weekplanner app. It exposes a REST-inspired API to the front end.
