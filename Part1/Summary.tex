\section{Summary}

This report describes the development process for team 11 in the GIRAF project of 2019. The GIRAF project develops tools for people the autism in collaboration with several institutions that work with autistic people who act as customers for the project. Sixth-semester Software students from AAU run the project. It has run since 2011, and each year the students continue where last year left off. Seven development teams are working together this year in the GIRAF project. The main focus of this report is the development process and how we refined it through the semester.

We describe two processes, the shared process of all development teams, and the process exclusive to team 11. The shared process was initially a Scrum of Scrums process with a scrum master group and a project owner group. However, both the planning and review activities were removed because they did not fit into our workflow. Also, the scrum master and project owner roles ended up as more of process and project manager groups than the traditional roles. 

The previous years, the development teams work on different pars of the system, server, frontend or backend, but this year we chose to work full-stack, where all teams worked on the whole system. We worked with meta groups for each part of the system, with one person from each team in each meta group. The meta groups made decisions about their part of the system, and made sure that every group had at least one member with knowledge about that part of the system 

During the semester we worked on one application called the Weekplanner that show the activities of people during a week. We did not work on any other applications. There were already a Weekplanner application made in Xamarin, but due to serious problems with getting it to work on Linux, and poor documentation, we made a new one from scratch. The new application was made in the Flutter framework, structured with the BLoC pattern. 

The work was done in four sprints, and at the end of each sprint we had a retrospect where we refined our process, both in team 11 and the shared process. The report describes the changes made in the retrospects.

The costumers were happy with the new application, as they found it more intuitive. There were, however, some features missing before it was useable. 

In the conclusion we discuss what we have learned about collaboration, both with many developers working on the same project, and teamwork in smaller groups. 
