\section{Summary}

This report describes the development process for team 11 in the \gls{giraf} project of 2019. The \gls{giraf} project develops tools for people the autism in collaboration with several institutions that work with autistic people who act as customers for the project. Sixth-semester Software students from Aalborg University run the project. The \gls{giraf} project started in 2011, and each year, the students continue where last year's students left off. Seven development teams are working together this year in the \gls{giraf} project. The main focus of this report is the development process and how we refined it through the semester.

We describe two processes, the shared process of all development teams, and the process exclusive to team 11. The shared process was initially a Scrum of Scrums process with a scrum master group and a project owner group. We removed both the planning and review activities as they did not fit into our workflow. Also, the scrum master and project owner roles ended up as more of process and project manager groups rather than the traditional roles. 

In the previous years, the development teams traditionally worked on separate parts of the system, server, frontend or backend, but this year we chose to work in a full-stack manner, where each team worked on parts across the whole system. We worked with meta groups for each part of the system, with one person from each team in each meta group. The \glspl{skillGroup} made decisions about their segment of the system and made sure that every group had at least one member with knowledge about that segment of the system. 

\gls{G19}'s focus in this semester was one application called the Weekplanner that show the activities of people during a week. We worked exclusively with this Weekplanner application and the backend. The Weekplanner application was built using the Xamarin framework, but due to some severe difficulties in compiling the application on Linux, we decided to rebuild the project with a new framework which supported compilation on Linux. We decided to build the application with the Flutter framework, structured with the BLoC pattern. 

In this report, we describe the four sprints we planned, and the retrospect we held at the end of each sprint, describing how we refined our process, both in team 11 and in the shared process.

The customers were happy with the new application, as they found it more intuitive. However, some features were missing before it was useable. 

In our conclusion, we discuss what we have learned about collaboration, both with many developers working on the same project, and teamwork in smaller groups. 
