\section{Introduction}

Many software projects require several people to participate in the development process. The engineers need to be able to work together and coordinate the work with each other, which can be complicated when the project gets large. It can be especially troubling if the development process is faulty or even worse if there is no development process.

No process fits all projects, so it makes sense to consider many factors like size, types of people, and the nature of the project. Even when considering all of these factors, there can still be aspects of the process which function suboptimally. Process models like \gls{Scrum} emphasizes the importance of reviewing and adapting the process \cite{Scrum}.

This report describes work on a project called \gls{giraf}. In this project, software students work together and consult with customers to develop a product. The students are divided into teams and define the work process themselves, continuing on the work from previous years. The main focus of this report is the development process.