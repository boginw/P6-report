\section{Introduction}

Many software projects require several people to participate in the development process. The engineers need to be able to work together and coordinate the work between each other. This can be complicated when the project get too large, especially if the development process is not defined or simply bad.

No process fits all projects and many factors like size, types of people and the nature of the project should be considered. Even when this is taken into account, there can be aspects of the process that does not work optimally. Process models like \gls{Scrum} empathies the importance of revising and adapting the process frequently\cite{Scrum}.

This project is called GIRAF. This has run for several years and is developed by sixth semester software students. After the semester the students hand over the project to the next students. This year seven \glspl{devTeam} with four to six members work together.

The main focus of this report is the development process and how it evolves during the semester. This includes both the process choices and the reasons behind them. We also describe the GIRAF process and how it evolves during the semester, with our work as the main focus.