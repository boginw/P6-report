\section{Group process}

In addition to the \gls{G19} process, our \gls{devTeam}, \gls{T11}, defined our internal process, which defines how we intend to work as a group and which practices we intend to uphold.

\subsection{XP}

We intend to work in the \gls{XP} spirit, with a focus on some of the practices and core values. The central practices we want to focus on are as follows:

\begin{itemize}
    \item Pair Programming
    \item Test-Driven Development
    \item Sustainable Pace
\end{itemize}

\subsubsection{Pair Programming}

It is essential for us that the implementations are of good quality, and the members of our team share knowledge. The pair programming is an attempt to increase quality because it makes it possible to get different thoughts on a solution and because the observer can review while the \gls{driver} writes the code. We also want to use pair programming to enable group members to learn from each other, as group members vary in skill and experience.

\subsubsection{Test-Driven Development}

Test-Drive Development relies on a short development cycle where the developer first writes a failing unit test which should explain the desired functionality, and then writes the production code to make that test pass. By utilizing a test-first driven approach, the developer is forced to imagine how to use the code before writing the code, which can lead to a more readable code-base. A good test-suite also yields more courage to change the code, as developers can verify that every change made does not break other code by running all of the tests.

\subsubsection{Sustainable Pace}

We find the sustainable pace practice critical to realize as all of our \gls{devTeam}'s members have had experience with having a too heavy workload, and therefore know the problems associated with such a workload firsthand.

For this practice, we have designed a schedule, where we calculate the number of hours used for project work for every week. This number corresponds to forty hours a week, minus the time dedicated to lectures. We then record how many hours each \gls{devTeam} member works each day, and make sure that they never work more than the forty hours in total per week. We will require our members to be at the university for the majority of the work hours so that we can help each other by for example doing pair programming, while each member decides for themselves where they will work for the rest of the hours, called flex-time. Flex-time is, e.g. staying home when writing report paragraphs.

\subsection{Additional Practices}

We will have a daily standup meeting, so everybody knows how the work is progressing, and at the end of every \gls{SOSSprintRetrospective}, we hold our retrospective, to improve our internal process. Here we will discuss issues, and attempt to make necessary changes to our process.

We agreed that the related documentation should always be produced alongside the event or work it reflects, which will ensure that we will not forget important details, or make knowledge implicit when it should not be.