\section{Group process}

In addition to the \gls{G19} process, our \gls{devTeam}, \gls{T11}, defined our own internal process.

This process defines how we intend to work as a group, and which practices we intent to uphold.

We intent to work in the \gls{XP} spirit, with focus on some of the practices and core values.
The central practices we want to focus on, are as follows:
\begin{itemize}
    \item Pair programming
    \item Test driven development
    \item Sustainable pace
\end{itemize}

It is important to us, that the work done is of good quality, and the members of our team share knowledge. The pair programming is meant to ensure quality, because it makes it possible to get different thoughts on a solution, and because the observer can review while the \gls{driver} writes the code. We also want to use pair programming to enable group members to learn from each other, as group members vary in skill and experience.

Test driven development is meant to ensure code quality because it makes sure that all code is tested, this makes it easier to find bugs, and works as documentation for the code.

We find the sustainable pace practice is especially important to uphold, because all of our team members have had experience with having a workload that is too big, and therefore know firsthand the problems associated with this.

For this practice, we have designed a schedule, where we calculate the amount of hours that should be used for project work every week. This corresponds to forty hours minus the time dedicated to lectures. We then note how many hours each day the different group members work, and make sure that they never work more than the forty hours in total. Some of the hours we work will be at the university, so we can help each other e.x. when doing pair programming, and work with other groups, and some hours will be up to each individual team member e.x. staying home when writing report paragraphs.

Apart from the \gls{XP} practices, we have also made some agreements.

One agreement is that the related documentation should always be produced alongside the event or work it reflects. This way we will not forget important details, or make knowledge implicit when it should not be.

We will have daily standup meetings, so everybody knows how the work is progressing.

In the end of every \gls{SOSSprintRetrospective}, we hold our own retrospective, to improve our internal process. Here we will discuss issues, and attempt to make necessary changes to our process.
