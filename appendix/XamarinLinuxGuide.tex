\chapter{Xamarin Linux Guide} \label{app:xamarin-linux}

\textbf{TODO}: ensure congruency between manual setup and install script

This documents outlines how to setup your environment such that you can
compile the \href{https://github.com/aau-giraf/weekplanner}{WeekPlanner}
app. To compile, you have four different options:

\begin{itemize}
    \item Using a Virtual Machine,
    \item using Docker,
    \item setting it up, manually,
    \item or using the install script.
\end{itemize}

Each of these options will be discussed in the following sections.

\textbf{NOTE}: This guide is licenced under
\href{https://opensource.org/licenses/MIT}{MIT} to protect the author's
ass.

\section{Virtual Machine}

To make things easy for everyone, a Virtual Machine is provided. It is
not recommended to use it if not necessary, as it is usually much slower
to do so. But in case you need to, you can download it
\href{https://drive.google.com/file/d/1fxhboUHWcESF-H_CMSkNJFypwQi8fvjI/view?usp=sharing}{here}.
Below are the credentials to the machine.

\begin{lstlisting}
username: aau
password: giraf
\end{lstlisting}

\section{Docker}

Just as the Virtual Machine, the Docker image comes preinstalled with
all the necessary dependancies to compile the project. Of course, to use
Docker you must first ensure that you have installed it. Given that you
have already installed Docker, \lstinline{cd} into your
project directory root, and execute the following, and a brand-new APK
file will appear.

\begin{lstlisting}[language=bash]
docker run -v "$(pwd):/weekplanner" -it aaugiraf/xamarin-android
\end{lstlisting}

If you want to ``ssh'' into the machine, you can attach to the
container's shell by executing the following:

\begin{lstlisting}[language=bash]
docker run -v "$(pwd):/weekplanner" -it aaugiraf/xamarin-android /bin/bash
\end{lstlisting}

For more information regarding the container, look
\href{https://cloud.docker.com/u/aaugiraf/repository/docker/aaugiraf/xamarin-android/general}{here}.

\section{Manually Setup}

To get started we first need root access, this is achieved by running:

\begin{lstlisting}[language=bash]
$ sudo su
\end{lstlisting}

Next, we choose which versions of the different things we're going to
use. We define the following variables for easy reference.

\begin{lstlisting}[language=bash]
DOTNET_SDK_VERSION=2.1.200
XAMARIN_VERSION=8.3.99.189
ANDROID_SDK_TOOLS_VERSION=3859397
\end{lstlisting}

Following this, we will install the general dependencies for Giraf's
Weekplanner.

\begin{lstlisting}[language=bash]
$ apt-get update
$ apt-get -y install \
             unzip \
             openjdk-8-jdk \ # only necessary if not already installed
             libzip4 \
             apt-transport-https \
             curl \
             gnupg
\end{lstlisting}

\subsection{Install Mono}

Installing mono requires adding to the sources list, this is done in the
following bash code:

\begin{lstlisting}[language=bash]
$ apt-key adv --keyserver hkp://keyserver.ubuntu.com:80 --recv-keys 3FA7E0328081BFF6A14DA29AA6A19B38D3D831EF
$ echo "deb https://download.mono-project.com/repo/ubuntu stable-bionic main" | tee /etc/apt/sources.list.d/mono-official-stable.list
$ apt-get update
$ apt-get install -y mono-devel
\end{lstlisting}

\subsection{Installing Android SDK}

To compile to Android, we need the Android SDK. The following will
download the SDK tools and SDK manager.

\begin{lstlisting}[language=bash]
$ curl https://dl.google.com/android/repository/sdk-tools-linux-${ANDROID_SDK_TOOLS_VERSION}.zip -o sdk-tools.zip
$ mkdir /android
$ unzip sdk-tools.zip -d /android
$ rm -rf sdk-tools.zip
$ rm -rf /android/licenses
\end{lstlisting}

We need to agree to some licenses, the following will update the Android
SDK manager, and agree to any licenses.

\begin{lstlisting}[language=bash]
$ yes | /android/tools/bin/sdkmanager --update --licenses
$ yes | /android/tools/bin/sdkmanager "platforms;android-26" "build-tools;27.0.3"
$ yes | /android/tools/bin/sdkmanager --update
\end{lstlisting}

Finally, we fetch the NDK-bundle and change the permission of the
\lstinline{/android/} directory to something more
accessable.

\begin{lstlisting}[language=bash]
$ /android/tools/bin/sdkmanager "ndk-bundle"
$ chmod -R 777 /android/
\end{lstlisting}

\subsection{Installing Xamarin}

Next, we need Xamarin.Android, which can be fetched directly from
Microsoft.

\begin{lstlisting}[language=bash]
$ curl https://xamjenkinsartifact.blob.core.windows.net/xamarin-android/xamarin-android/xamarin.android-oss_${XAMARIN_VERSION}.orig.tar.bz2 -o xamarin.tar.bz2
$ tar xvjf xamarin.tar.bz2
$ mv xamarin.android-oss_* /android/xamarin
$ rm -rf xamarin.tar.bz2
\end{lstlisting}

\subsection{Installing DotNet}

\begin{lstlisting}[language=bash]
$ curl https://packages.microsoft.com/keys/microsoft.asc | gpg --dearmor > microsoft.gpg
$ mv microsoft.gpg /etc/apt/trusted.gpg.d/microsoft.gpg
$ sh -c 'echo "deb [arch=amd64] https://packages.microsoft.com/repos/microsoft-ubuntu-bionic-prod bionic main" > /etc/apt/sources.list.d/dotnetdev.list'
$ apt-get update
$ apt-get install -y dotnet-sdk-${DOTNET_SDK_VERSION}
\end{lstlisting}

\begin{lstlisting}[language=bash]
$ curl https://dist.nuget.org/win-x86-commandline/latest/nuget.exe -o /nuget.exe
$ cp -r /android/xamarin/bin/Release/lib/xamarin.android/* /usr/lib/mono/
$ mkdir -p /usr/lib/mono/xamarin-android/bin/
$ cp -r /android/xamarin/bin/Release/lib/xamarin.android/xbuild/* /usr/share/dotnet/sdk/${DOTNET_SDK_VERSION}
\end{lstlisting}

\subsection{Full script}

The below script was tested on a blank installation of Ubuntu 18.04.2,
and ran successfully.

\begin{lstlisting}[language=bash]
#!/bin/bash
if [ "$EUID" -ne 0 ]
  then echo "Please run as root"
  exit 1
fi

DOTNET_SDK_VERSION=2.1.200
XAMARIN_VERSION=8.3.99.189
ANDROID_SDK_TOOLS_VERSION=3859397

apt-get update
apt-get -y install unzip openjdk-8-jdk libzip4 apt-transport-https curl gnupg

# Set default java version to Java 8
update-java-alternatives --set java-1.8.0-openjdk-amd64

# Ignore saying yes to stuff
DEBIAN_FRONTEND=noninteractive

# Mono
apt-key adv --keyserver hkp://keyserver.ubuntu.com:80 --recv-keys 3FA7E0328081BFF6A14DA29AA6A19B38D3D831EF
echo "deb https://download.mono-project.com/repo/ubuntu stable-bionic main" | tee /etc/apt/sources.list.d/mono-official-stable.list
apt-get update
apt-get install -y mono-devel mono-complete

# Android
curl https://dl.google.com/android/repository/sdk-tools-linux-${ANDROID_SDK_TOOLS_VERSION}.zip -o sdk-tools.zip
mkdir /android
unzip sdk-tools.zip -d /android
rm -rf sdk-tools.zip
rm -rf /android/licenses
yes | /android/tools/bin/sdkmanager --update --licenses
yes | /android/tools/bin/sdkmanager "platforms;android-26" "build-tools;27.0.3"
yes | /android/tools/bin/sdkmanager --update
/android/tools/bin/sdkmanager "ndk-bundle"
chmod -R 777 /android/

# Xamarin
curl https://xamjenkinsartifact.blob.core.windows.net/xamarin-android/xamarin-android/xamarin.android-oss_${XAMARIN_VERSION}.orig.tar.bz2 -o xamarin.tar.bz2
tar xvjf xamarin.tar.bz2
mv xamarin.android-oss_* /android/xamarin
rm -rf xamarin.tar.bz2

# Dotnet
curl https://packages.microsoft.com/keys/microsoft.asc | gpg --dearmor > microsoft.gpg
mv microsoft.gpg /etc/apt/trusted.gpg.d/microsoft.gpg
sh -c 'echo "deb [arch=amd64] https://packages.microsoft.com/repos/microsoft-ubuntu-bionic-prod bionic main" > /etc/apt/sources.list.d/dotnetdev.list'
apt-get update
apt-get install -y dotnet-sdk-${DOTNET_SDK_VERSION}

printf "\nexport ANDROID_SDK_PATH=/android/" >> /etc/environment
printf "\nexport DOTNET_CLI_TELEMETRY_OPTOUT=1" >> /etc/environment
printf "\nexport MSBuildSDKsPath=/usr/share/dotnet/sdk/${DOTNET_SDK_VERSION}" >> /etc/environment

curl https://dist.nuget.org/win-x86-commandline/latest/nuget.exe -o /nuget.exe
cp -r /android/xamarin/bin/Release/lib/xamarin.android/* /usr/lib/mono/
mkdir -p /usr/lib/mono/xamarin-android/bin/
cp -r /android/xamarin/bin/Release/lib/xamarin.android/xbuild/* /usr/share/dotnet/sdk/${DOTNET_SDK_VERSION}
\end{lstlisting}

\section{Common problems}

\subsection{Error \#1}

\begin{lstlisting}[language=bash]
System.TypeInitializationException: The type initializer for 'System.Console' threw an exception. ---> System.TypeInitializationException: The type initializer for 'System.ConsoleDriver' threw an exception. ---> System.Exception: Magic number is wrong: 542
\end{lstlisting}

\begin{lstlisting}[language=bash]
TERM=xterm
\end{lstlisting}

\subsection{Error \#2}

NET Framework 4.5

\begin{lstlisting}
printf "\nexport FrameworkPathOverride=/usr/lib/mono/4.5/" >> /etc/environment
\end{lstlisting}

