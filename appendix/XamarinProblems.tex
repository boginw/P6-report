\chapter{Xamarin Problems} \label{app:xamarin-probelms}

This document will describe why a switch of framework should be considered, and what work lies ahead if a switch is chosen. Keep in mind, that the author's bias towards switching.

\section{Problems with current solution}

Right of the bat, I assume that most of the readers, who run Linux, started by following JetBrain's \href{https://rider-support.jetbrains.com/hc/en-us/articles/360000557259}{How to develop Xamarin.Android applications on Linux with Rider} guide, and were dissapointed to learn that this guide does not lead to a successful compilation of the WeekPlanner app. And as can also be learned by said guide:

\begin{quote}
    Please note that Xamarin.Android on Linux is officially unsupported
\end{quote}

This unfortunate fact, means that approximately a quarter of the developers cannot get started developing without additional hassle.

Without looking at alternative solutions, but using the \href{https://github.com/aau-giraf/xamarin-android/blob/master/Dockerfile}{Dockerfile}, created from last-semester's Docker Image, the following problems were
identified.

\begin{itemize}
    \item Old version of DotNetCore
    \item Non-obvious Xamarin.Android download
    \item Non-standard installation nuget
    \item Complex folder moving and copying
\end{itemize}

It should be noted, that one \emph{can} get compilation to work on
Linux, albeit a hassle. Additional problems occur thereafter, when the
developer tries to compile the Web-Api which no longer works. Following
last-semester's implementation, Linux users, must choose if they want
Web-Api or WeekPlanner to compile.

It should again be noted that one \emph{can} get compilation of the
application and Web-Api to work, by using an IDE that supports multiple
DotNetCore versions.

In my opinion, there are too many loops to go through to achieve an
environment that can compile everything, and if we are to ease the pain
of the comming semester something needs to be done.

\subsection{Additional Problems}

A newly discovered issue lies within the generated code
\href{https://github.com/aau-giraf/weekplanner/tree/master/IO.Swagger}{IO.Swagger}.
IO.Swagger works when downloaded from the repository, but in trying to
regenereate IO.Swagger from a specification file, or from
\texttt{update\_swagger.sh}, results in multiple errors. These errors
are tightly coupled with the Weekplanner project, and as so, are
probably easy fixes.

\section{Solutions}

Some effort has been put in creating an install script (see
\href{https://hackmd.io/_gtHEeO6Q36ww-oANJrJaw}{this}), which can be
extended to get Web-Api to also work. The downside of this solution, is
that the script needs to be maintained, and updating it \emph{might}
result in someone needing to reinstall his/hers OS inorder to get it to
work.

Alternatively, a Dockerised solution can be used, but this solution has
to be created, and set up in a way, that makes compilation, debugging,
and deployment as easy as with other OSes. I personally don't see this
being done in the near future, as this aswell requires major work.

Finally, the nuclear option, switch framework. If we were to switch to
an alternative framework, which supports Windows, MacOS, and Linux, then
everyone will be happy, and there should be no reasonable reason for
future semesters to switch.

\section{Advantages to switching}

There are also many minor advantages of switching, i.e live reloading,
but constraining it to the crux of the problem, the following advantages
were found.

The main issue, that is to be solved by switching framework, is
cross-platform support. It should not be a hassle for any contributor to
get started. In switching to \href{https://flutter.dev/}{Flutter}, this
can be achieved.

Contrary to the current tests, of which there are few, and even fewer
that work, Flutter comes packaged with a testing environment.

Flutter also allows for fixing the fundamental error in the current
structure, of using generated code. The error does not lie in using
generated code, as this can be great when done right, but in the trust
that the generator, or the format it understands, will be supported for
the comming semesters.

\section{Disadvantages to switching}

The biggest disadvantage is reimplentation, as all current code needs to
be ported. Fortunately, this is not too much.

Another disadvantage, if Flutter is chosen as the alternative, is the
paradigm of Flutter. Flutter is OOP but relies on the understanding of
state- or state-less widgets.

Regarding architecture support of Flutter,
\href{https://github.com/flutter/flutter/issues/14925}{it does not
support 32 bit}, which might be a problem, for people with very old
computers.

\section{FAQ}

\begin{quote}
Why not just switch to Windows?
\end{quote}

There are two parts to this answer: 1. Due to the question, I assume
that you run Windows. Would you think it is fair to ask of me in an
alternate setting to switch to Linux? 2. There are many different
reasons people run Linux, one of them is because of Hardware. Not all
computers \textbf{can} run Windows.

\begin{quote}
If we choose to switch, then a LOT of work will go to re-implement
something that already is done. Isn't this a major waste of time?
\end{quote}

There are two sides to this answer. Currently, the app is in a poor state. There is
MAJOR work ahead, to even just get it stable. It is fortunate that the
functionality of the app is very limited, so it is not unreasonable that
the rewrite can be done in a timely manner.

\begin{quote}
But if it \emph{can} work on Linux, why then change?
\end{quote}

The point is not that it cannot be made to work, the point is how many
hoops we have to jump through in order to get it to work. Not everyone
running Linux is a super-nerd that is comfortable playing around with
the terminal and so on. I think it is unreasonable to make them go
through these hoops.

\begin{quote}
Can't the Linux people just work on something else than front-end?
\end{quote}

Yes, but that is not a solution. This excludes 1/4 of the people from
working front-end, which seems really unfair.

\begin{quote}
What about dual-booting Windows?
\end{quote}

By the same argument for why not run Windows full-time, this is not a
viable argument. But also, one might not have disk space to create a
partition for Windows.

\begin{quote}
What is to stop future semesters from switching again, and effectively
nullify our work?
\end{quote}

Look at this way; there are many here who hate C\# and DotNet, but none
of them suggest switching the framework of the Web-Api. That's because
there is no reasonable reason to do so. It is supported on all
platforms, and runs perfectly fine everywhere. The same cannot be said
of the app, but if it were, I believe that by the same reasoning, future
semester would not switch either.

\begin{quote}
If we switch then there is nothing to write about in the report because
everyone needs to work on front-end
\end{quote}

Since our focus this semester was to get the app stable, there should
probably not be too much work in back-end, and server is limited
anyways. And given that there are major problems with the generated
Swagger API, we're in for a major rewrite anyways. Switch or not, I see
most people working in front-end regardless.

\section{Conclusion}

Given the unsupporting nature of Xamarin, a switch is inevitable, it is
only a question of when. I propose ripping the bandaid off, and build a
core, that can be extended by everyone.

\section{Switching}

In case that a switch is decided, then there is some work ahead. To
understand what is to be done, the following two sections will describe
what pages exist currently, and related functionality.

\section{Pages to be implemented}

There are 11/14 pages that need to be re-implemented, in the table
below, the 14 pages are listed and their purpose described.

\begin{longtable}[]{@{}ll@{}}
\toprule
Page name & Description\tabularnewline
\midrule
\endhead
ActivityPage & Info page for Activity\tabularnewline
ChoiceBoardPage ~ & Page for choosing between activities\tabularnewline
ChooseCitizenPage & Vælg Borger\tabularnewline
ChooseTemplatePage & Choose Week Template to use/edit\tabularnewline
CitizenSchedulesPage & Vælg Skema\tabularnewline
CustomNavigationPage & Wrapper to achieve navigation\tabularnewline
LoginPage & Page for logging a user in\tabularnewline
MasterPage & NA\tabularnewline
NewSchedulePage & Tilføj en ny ugepplan\tabularnewline
PictogramSearchPage & Where you search for Pictogram\tabularnewline
SavePromptPage & (Not Used)\tabularnewline
SettingsPage & Page where all settings are listed\tabularnewline
TestingPage & Page for testing views (trivial)\tabularnewline
WeekPlannerPage & Crux of the app, the week overview\tabularnewline
WeekPlannerTemplatePage & Create/Edit Week template\tabularnewline
\bottomrule
\end{longtable}

\section{Other functionality}

\begin{longtable}[]{@{}ll@{}}
\toprule
Category & Description\tabularnewline
\midrule
\endhead
ApplicationObjects & Dependency Injection\tabularnewline
Behaviors & Describe how views can change, e.g select\tabularnewline
Controls & To be used by views\tabularnewline
Converters & To be used by views\tabularnewline
Helpers & Colleciton of functions grouped into classes\tabularnewline
Models & Simple objects that should mainly hold data\tabularnewline
Services & Login, Navigation, Request, and Settings\tabularnewline
Themes & Alternative visible stuff\tabularnewline
Validations & Validation for forms\tabularnewline
ViewModels & Code for the visual stuff\tabularnewline
Views & Visible stuff\tabularnewline
\bottomrule
\end{longtable}

\section{Re-implementation plan}

\begin{itemize}
    \item Core Implementation of the Application in Flutter
    \item Coding standards
\end{itemize}
