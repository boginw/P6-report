\section{SOS retrospective}

The last \gls{SOS} activity in sprint 1 was the retrospective where all the members of \gls{giraf} attended. At the meeting, we formed subgroups that consisted of at most one member from every team. 

The subgroups then had fifteen minutes to find ideas for changes to the process. After that, the \gls{ST} reviewed the ideas and removed duplicates. Everybody could then vote for the three most essential ideas in their perspective. After this, the \gls{ST} took the feedback and made changes accordingly. Most members were satisfied with the overall process and the way the groups communicated.

The change to Flutter, however, made it hard to decide if the process structure was successful because the process flow got broken during the hasty move to Flutter. Therefore, many practices were never really used, and there was not a release ready application.

One change made to the process was, that stand up meetings should be on days with lectures, because some groups worked predominantly from home, so it would suit them better if the meetings were placed close to lectures, where they would already be present. 

The \gls{ST} also changed the way sprint planning is carried out. Rather than giving the groups all the tasks estimated for the sprint up front, they now get to take one from the backlog when needed. The sprint planning is then more of a sprint intro meeting where the \gls{PO} informs the \glspl{devTeam} of the purpose of the sprint. 
