\section{Moving The Frontend}

While setting up for development, we became aware of the hassle in order to get Xamarin to work on the operating system Linux. We created a guide which should encompass all the different quirks and workaround for the different errors (see \autoref{app:xamarin-linux}). The guide has four ways of compiling Xamarin on Linux: Using a Virtual Machine, using Docker, or manual installation. At one point we noticed, that by using the guide the ability to compile the \gls{api}, which is also compiled using C\#, diminished. At this point, the guide had grown more extensive than expected, and the future for Linux and Xamarin seemed dim.

We conducted an informal count and found that one in four students in the project used Linux, and we have no reason to think that there will be fewer in the future. This finding solidified our concerns, and we called for an emergency meeting with the front-end skill group.

At the meeting, we discussed all the concerns with the existing solution, and at the end, we agreed, that a switch from Xamarin to another development framework would be best for the project. We discussed React Native\cite{react-native:website}, Cordova\cite{cordova:website}, and Flutter\cite{flutter:website}, but in the end, we chose Flutter as it stood as the strongest contestant. For a more detailed description of the issues discussed see \autoref{app:xamarin-probelms}.

Unfortunately, the new development framework meant that we needed to rewrite the entire application. In the following sections, we describe what Flutter is, the architectural pattern used, and how we migrated.
