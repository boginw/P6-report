\section{Moving The Frontend}\label{sec:MovingTheFrontend}

When we began development on the Weekplanner app, we discovered that it was challenging to get Xamarin to work on the operating system Linux . We created a guide which should encompass all the different quirks and workaround for the different errors (see \autoref{app:xamarin-linux}), to help other groups getting Xamarin to work. The guide describes three ways of compiling Xamarin on Linux: Using a Virtual Machine, using Docker, or a manual installation. At one point, we noticed, that by using the guide the ability to compile the \gls{api}, which is also compiled using C\#, diminished. At this point, the guide had grown more extensive than expected because Xamarin had to be installed differently on different Linux versions. We concluded that it was unrealistic to make a guide that would work on the most current and future versions.

We conducted an informal count and found that one in four students in the project used Linux, and we have no reason to think that there will be fewer in the future. This finding solidified our concerns, and we called for an emergency meeting with the front-end skill group.

At the meeting, we discussed all the concerns with the existing solution, and at the end, we agreed that a switch from Xamarin to another development framework would be best for the project. We chose a framework called Flutter\cite{flutter:website}, as this worked on both Mac, Windows, and Linux computers and automatically compiled to both Android and iOS. For a more detailed description of the issues discussed, see \autoref{app:xamarin-probelms}.

The new development framework meant that we needed to rewrite the entire application. In the following sections, we describe what Flutter is, the architectural pattern used, and how we migrated.
