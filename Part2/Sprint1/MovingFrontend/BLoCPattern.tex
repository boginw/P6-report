\section{BLoC Pattern}
When we moved the application to Flutter, we chose to structure the code with the \gls{bloc} pattern. The following chapter describes this pattern. 

\subsection{Background}
\gls{bloc} is a pattern that aims to isolate all business logic from \gls{ui} code. Two software engineers, Paolo Soares and Cong Hui, designed the pattern while they worked on the Google platform; AdWords. Paolo Soares presented the \gls{bloc} pattern at DartConf 2018 in a speach called "Code sharing, better together". 
The background for the \gls{bloc} pattern is that the Google AdWords team  developed two applications; a web-application and a mobile application. 
They developed the applications before Flutter existed, and had to make native code bases, for iOS and Android, for the mobile application, as well as one for the web application. 
Because they had three separate code bases, they had to maintain all three bases and repeat all changes three times. 

When Flutter was released Paolo Soares, and Cong Hui saw the possibility to merge the mobile-applications into one code base and thereby to achieve only one language for the development. They also initially thought that they would be able to reuse all the dart written business logic from the web-application and thereby very rapidly develop the mobile application. Trying to do so highlighted a considerable issue in their web-application which was that the business logic where mixed in with the \gls{ui} logic.

So, if they wanted to re-use the business logic from the web-application they had first to clean up the entire application to separate business and \gls{ui} logic, but not only should it be separated, but the business logic should also be platform independent meaning that it should be useable without any modifications on both the web-application and the mobile-application. The pattern they ended up with as a solution for this was named \gls{bloc}.

The pattern alone does not ensure separated business and \gls{ui} logic, so they created a list of rules that has to be upheld by all developers that uses the pattern without exceptions.

\subsection{What is BLoC}
The \gls{bloc} pattern is merely a pattern that enforces all business logic to lay outside the \gls{ui} components and communicate via streams. The communication of a \gls{bloc} is heavily inspired by the Observer pattern and Iterator Pattern and thereby bares many similarities with the ReactiveX API. In fact, in the Giraf project, the \gls{bloc} pattern is implemented using the ReactiveX API via a package called rxDart. The ReactiveX API describe them self as "ReactiveX is a combination of the best ideas from the Observer pattern, the Iterator pattern, and functional programming" \cite{ReactiveXWebsite}.

\subsection{BLoC Pattern Rules}
In the talk from DartConf 2018, Paolo Soares suggested a set of rules or best practices that developers should uphold when they use the \gls{bloc} pattern. Here follows an introduction to the rules suggested by Paolo Soares:

\subsubsection{BLoC Design Guidelines}
\begin{itemize}
  \item Inputs and outputs are simple Streams/Sinks only
  \item Dependencies must be injectable and platform agnostic
  \item No platform branching allowed
  \item Implementation can be whatever you want if you follow the previous rules
\end{itemize}

The first rule states that any communication between the \gls{ui} and a \gls{bloc} must be through Sinks or Streams, so when the \gls{ui} should send data to the \gls{bloc} it should be a sink, and when a \gls{bloc} should send data to the \gls{ui} it should be a stream.

The second rule states that all dependency of a \gls{bloc} must be injectable; this is a crucial step in reaching platform independence and does not require much explanation.

The third rule states that one is not allowed to branch depending on the platform, meaning the business logic should be completely independent of the platform. What the rule is effectively stating is that inside a \gls{bloc} there should not be e.x., an if-statement checking if the platform is iOS and do some computation on behave of that.

The last and fourth rule can be a bit hard to understand, but what it states is that one can implement the \gls{bloc} pattern however they like, i.e., like in the Giraf Project using rxDart and thereby reactive programming.

\subsubsection{UI Design Guidelines}
\begin{itemize}
  \item Each "Complex enough" component has a corresponding \gls{bloc}
  \item Components should send inputs "as is"
  \item Components should show outputs as close as possible to "as is"
  \item All branching should be based on simple \gls{bloc} Boolean outputs
\end{itemize}

The first rule states that if the \gls{ui} component is complex enough it should have its own \gls{bloc}, thereby also stating that \gls{ui} components should share \glspl{bloc} if they are non-complex.

The second rule states that if the \gls{ui} needs to send data to the \gls{bloc}, i.e., User inputs, the \gls{ui} component is not allowed to do any computation on the inputs before sending it to the \gls{bloc}; this is a step to ensure decoupling between the \gls{ui} and Business logic.

The third rule is the same as the second rule but in the other direction. So the output from a \gls{bloc} should not be changed inside the \gls{ui} component, this again is to ensure that all business logic stays inside the \gls{bloc}.

The fourth rule states that all if statements inside the \gls{ui} component should only depend on one boolean stream from the \gls{bloc}, this again is to decouple the \gls{ui} component from the business logic, since branching in the \gls{ui} most likely are based on some business logic.

\subsection{BLoC Design Guidelines of the Giraf Project}
As mentioned earlier the Giraf Project have altered the rules of the \gls{bloc} pattern a bit. The reason for this is that in the Giraf Project we are not using the code in other platforms than the Flutter Framework, therefore to make the use of \glspl{bloc} a bit more intuitive and flexible for the developers we have constructed the following design guideline:

\subsubsection{Giraf Project BLoC Pattern Guidelines}
\begin{itemize}
  \item Rather many small non-complex \glspl{bloc} than few large complex \glspl{bloc}
  \item Start by creating a \gls{bloc} per \gls{ui} Screen, then after implementing the functionality consider to refactor to shared \glspl{bloc}
  \item Inputs to \glspl{bloc} should be either Sinks or function calls with parameters
  \item \glspl{bloc} should be instantiated via the dependency injector/ Thereby all dependencies should also be injectable
  \item \glspl{bloc} are to be implemented using the rxDart library
\end{itemize}


