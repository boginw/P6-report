\section{SOS Sprint Review}
In the end of sprint 1 \gls{G19} held a review meeting. This section describes that meeting.
The \gls{G19} \glspl{devTeam} started with features for the Xamarin implemented app. However, the Xamarin ecosystem was disallowing a lot of the developers to work since the Xamarin project has abandoned Linux support. Since there were roughly 40\% of the developers on the GIRAF project running Linux, this was a significant issue.

Due to the Linux issue of Xamarin, the front-end meta group decided that a move to Flutter was necessary. At the end of Sprint 1 the transition from Xamarin to Flutter where not at a release ready state. The status of the new Flutter app was that the architecture, data models, API module and some initial demo-screens where implemented.

Within Sprint 1 the first alpha version of the Flutter app core where released to the \glspl{devTeam} and the \gls{PO} group had prepared new user stories for the \glspl{devTeam} to tackle. The developers behind the alpha core held a release presentation where all developers of the \gls{G19} where present. This presentation was an introduction to Flutter, the architecture of the new core and a coding session where a user story where implemented. Afterward, the remaining user stories were delegated to each \gls{devTeam}, leaving roughly one week before the end of the sprint. The short timeline ment that most \gls{devTeam} were not  able to finish their story before the end of the sprint. Never the less the overall feedback was that the developers were happy with the transition to Flutter, they felt that the development where more rapid and the UI was more comfortable to implement. Most \glspl{devTeam} nearly finished with their stories, but most of them were blocked by lacks in the core mainly: The ability to transmit data from one component to another, The ability to implement widget tests, and a bug in the authentication system.

The \gls{PO} group where satisfied with the progress done this sprint and considered the move to Flutter an elimination of serious technical debt and thereby an improvement to the quality of the product.
