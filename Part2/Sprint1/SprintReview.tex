\section{Review}

The \gls{G19} \glspl{devTeam} started with features for the Xamarin implemented app. However, the Xamarin framework hindered a lot of the developers, since the Xamarin project has abandoned Linux support. Since there was roughly one-fourth of the developers on the \gls{giraf} project running Linux, this was a significant issue.

Due to the Linux issue of Xamarin, the front-end meta group decided that a move to Flutter was necessary. At the end of sprint 1 the transition from Xamarin to Flutter was not ready for the release but we had implemented the architecture, data models, \gls{fapi} module, and some initial demo-screens.

Within sprint 1, the first alpha version of the Flutter application core was released to the \glspl{devTeam} and the \gls{PO} group had prepared new user stories for the \glspl{devTeam} to tackle. The developers behind the alpha core held a release presentation where all developers of the \gls{G19} where present. This presentation was an introduction to Flutter, the architecture of the new core and a coding session where we, \gls{T11}, implemented a user story. Afterward, the remaining user stories were delegated to each \gls{devTeam}, leaving roughly one week before the end of the sprint. The short timeline meant that most \glspl{devTeam} were not able to finish their story before the end of the sprint.

Nevertheless, the overall feedback was that the developers were happy with the transition to Flutter, they felt that the development was more rapid, and the \gls{ui} was more comfortable to implement. Most \glspl{devTeam} nearly finished their user-stories, but shortcomings in the core blocked most of them. The most significant blockers were the missing ability to transmit data from one component to another, the ability to implement widget tests, and a bug in the authentication system.

The \gls{PO} group was satisfied with the progress made this sprint and considered the move to Flutter an elimination of severe technical debt and thereby an improvement to the quality of the product.
