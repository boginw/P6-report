\section{Sprint Goal}

Each sprint has a corresponding set of goals. Before the sprint, the \gls{POT} had been in contact with the customers. They all ranked a more stable Weekplanner the most crucial thing for them.

At this stage, the Weekplanner application lacked both functionality and stability. The customers also found the \gls{ui} to be confusing and some times, inconsistent. The customer also wanted offline functionality, primarily because they do not have stable wireless coverage, and often go on trips to places without internet.

The sprint started with a sprint planning with all of the members of \gls{G19} present. The \gls{POT} chose to update the Weekplanner application as the sprint focus, based on the costumers' wishes. The \gls{POT} presented the user-stories in prioritized order, then gave the groups time to discuss the different stories internally. After discussing the stories the groups then had three votes, all the groups then voted on the stories they wanted, and if there were too many groups requesting the same user story, we negotiated in order to make a fair allocation of user-stories. The voting was intended to delegate the work to the groups who found it interesting rather than with a first-come-first-served approach. \gls{T11} was assigned the following three user stories.

\begin{description}
    \item [\#14] As a citizen, I would like the icons to be consistent throughout the system so that I instinctively know their meaning.
    \item [\#15] As a citizen, I would like to be able to choose how many days I see at a time on my Weekplanner so that it fits my personal preference.
    \item [\#19] As a user, I would like the icons to be updated so that they are modern and easy to understand.
\end{description}

\section{Updating the icons}

Issue \#14 describes the issue of unclear icon meanings. That is, the icons have multiple meanings depending on where in the application they are used. The users have expressed concern regarding the simplicity and intuitiveness of the application and would prefer if the application was intuitive enough so that training is unnecessary. As a part of this issue, all icons and their meanings will have to be mapped, to see which icons make sense, and as far as possible only have one meaning. 

As \#19 is closely related to \#14, it makes sense to solve \#19 in tandem with \#14, as while we map the icons and their meanings, updating them alongside reduces some redundant work.

The \gls{PO} group made a design guide in cooperation with the customers. The assignment will then be to make the Weekplanner application respect this design guide.

\section{Changing how many days the Weekplanner shows} \label{sec:weekPlannerDaysToShow}

Issue \#15 expresses a user need for being able to change how many days the Weekplanner shows at a time. This ability is a need expressed by some citizen, as they can feel a bit overwhelmed by all the tasks of multiple days, so they would like to be able to show fewer days, or even just one.
