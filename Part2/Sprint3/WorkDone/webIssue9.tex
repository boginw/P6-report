\section{Fetching Week-Plans for Citizens} \label{sec:webIssue9}

One of the tasks in sprint 3 was to implement an endpoint for retrieving all the week-plans for a \gls{citizen}. 

The problem with the existing endpoint for fetching the week-plans for a specific \gls{citizen} is that it only returns all the names of the week-plans belonging to a \gls{citizen} and not other data such as the image for the week-plan. To get a complete list of week-plan objects for a \gls{citizen} it is required to iterate over all the week-plan names to request its full object from the \gls{api}, instead of having an endpoint which can do this in a single request. We considered the following three ways of solutions:

\begin{itemize}
    \item The quickest solution is to adjust the endpoint to return the correct object, which would also remove the need for a new endpoint. However, this change would result in all the frontend code, relying on this endpoint would break.
    \item Another way is to make a completely new endpoint, which would not break the frontend, but it would deviate from the way we have made the endpoints for the other resources.
    \item The last option was to create a new version of the \gls{api} with the abovementioned endpoint updated to return complete week-plan objects, which would allow for the old one to exist along with the new version. This way, the frontend would not break, and the \gls{api} would keep its interface stability.
\end{itemize}

We chose the last option as it seemed like the option with the least downsides, and the solution would ensure the integrity of the \gls{api} interface. It does add overhead when handling different versions of the \gls{api}.
