\section{API issue 9: New Endpoint} \label{sec:webIssue9}

One of the tasks in sprint 3 was to implement or modify an endpoint for retrieving all the weekdays of a citizen. The problem with the existing endpoint is if one wants to retrieve the weekplans, it only returns all the names of the weekplans belonging to a citizen thereby missing other data such as the image for the weekplan. This means that in order to get a complete weekplan object, it is necessary to iterate over all the weekplan names and make a new request to the api for each name to get the whole weekplan object, instead of having an endpoint which can do this in a single request. 
The task in itself is rather trivial, but there are some interesting considerations to be made before implementing it. The previous Giraf project teams implemented a REST \gls{api} which is a \gls{api} design philosophy so the new endpoint should preferably not break the REST principles. Though the \gls{api} did not follow REST completely in this endpoint, as the endpoint return different object types depending on which HTTP request was made. 

We considered the following three ways of solving this issue:
\begin{itemize}
\item The quickest solution is to just adjust the endpoint to return the correct object, this would also remove the need for a new endpoint. But this change would result in all the frontend code relying on this endpoint would break until a group adopted the frontend to the change.
\item Another way is to make a new endpoint for implementing this change. This would not break the frontend, but it would deviate from the way we have made the endpoints for the other resources. As they generally have five actions per resource: \textit{get a collection, get a member, create a member, update a member, delete a member},  in correlation to the REST philosophy but this change would make a sixth necessary, that would be a new \textit{get a member} which would break the REST design philosophy, and make this endpoint inconsistent with the others.
\item The last option was to update the current endpoint, but instead of overriding it we created a new version of the entire \gls{api}, i.e. starting a v.2.0, and implement the modified endpoint in that. That would allow for the old one still to exist along with the new version. This way one can stay true to the REST convention of having interface stability, and still not break the frontend temporarily.
\end{itemize}

We chose the last option as it seemed like the option with the least downsides, and the solution would still comply with the REST philosophy. It does add overhead when handling different versions of the endpoints, which the frontend was not fully equipped for this.
