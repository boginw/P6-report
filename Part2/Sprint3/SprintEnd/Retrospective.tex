\section{SOS Retrospective}

The \gls{ST} decided to do the retrospective a little different this time. Instead of the dotstorming website, the \gls{ST} divided us into groups as usual and discussed the process with the \glspl{devTeam}. They noted the different surgestions, and used these notes to make a questionair. 

Everybody answered the questions with either aggre, don't care or disagree. 

The \gls{ST} evaluated the questionairs, and came to the following conclusions. 

\begin{itemize}
    \item Stand up meetings should focus more on explaining what the user stories different goups work on means. This will make it easier to find dependencies between groups. 
    \item If you work on a user storie and find out it is blocked, you should note what blocks the story, and remove yourself from it so it does not appear like you are working on it.
    \item Each user story should have one \gls{PR}, unless two user stories depend on each other. 
    \item You should read through all of the \gls{PR} whey you review it.
    \item If possible, a \gls{PR} should be reviewed the same day you are assigned to it. If this is not possible, you should contact the owner of the \gls{PR}
    \item Be more active on Slack, look at least once a day.
    \item Contact \gls{POT} if you do not have anything to do.
    \item Tell \gls{POT} if your user story can not be finished in the current sprint. 
\end{itemize}

These changes are implemented in sprint 4. 