\section{SOS Retrospective}

With the feedback noted, instead of the \glspl{devTeam} voting, this time, the \gls{ST} had us fill out a questionnaire created from the feedback. Everybody answered the questions with either agree, do not care, or disagree. The \gls{ST} evaluated the questionnaires and came to the following conclusions, which were implemented in sprint 4:

\begin{itemize}
    \item Stand up meetings should focus more on explaining the meaning behind the user stories on which the different groups work, which makes it easier to find dependencies between groups. 
    \item If a developer works on a user story and finds that said user story is blocked, they should note what blocks the story, and unassign themselves from it, so it does not appear like they are working on it.
    \item Each user story should have one \gls{PR} unless two user stories depend on each other. 
    \item You should read through the entire \gls{PR} whey you review it.
    \item If possible, a \gls{PR} should be reviewed the same day as assigned. If not possible, you should contact the author of the \gls{PR}.
    \item Be more active on Slack, check it at least once a day.
    \item Contact \gls{POT} if you do not have anything to do.
    \item Tell \gls{POT} if your user story cannot be finished in time.
\end{itemize}
