\section{Disappearing Activities}

Previously, we helped groups fixing an issue where if we added an activity to a week-plan, it would disappear if we exited the week-plan to reenter again. This issue was initially resolved in sprint three but had reappeared in sprint four. We choose to take a look at why this had happened and fix it yet again.

One of the benefits of using a version control system like Git is that for each change we check-in, a new version is created and stored. Each version has a set of changes and an associated description. We identified the commit where the issue was resolved initially \cite{weekplan_issue_257_fix}. This commit was committed directly onto the master branch. The commit immediately following this commit \cite{weekplan_issue_257_fix-2}  overwrote the fix, as if it was never there. We noticed that the second commit had been merged into the branch develop first, and then master. We speculate that because of the fixing commit was directly on the master branch, and the second was onto develop, that when the develop branch eventually got merged with master, the merge conflict was either overwritten or nonexistent. Either way, we had to fix the issue again. 

After consulting the group who had previously fixed the issue, we quickly solved the issue. When loading the overview of week-plans, all of its activities were also loaded and passed along to the week-plan screen, which meant, the week-plan screen displayed the activities from the overview when it was loaded. The back-end confirmed the addition of activities when we added activities to the week-plan, but the overview still had an old version of the activities, which meant that going back to the overview, selecting the same week-plan would load the activities which were present before we added additional activities. To solve the issue, we ensured that before rendering week-plans' activities, its activities were re-loaded before anything was displayed.

It is clear that overwriting already fixed issues is not a suitable procedure, and as such, we suggest implementing a prevention strategy for this error. We suggest that all reviewers check the base for all pull-requests before approving as to ensure that nobody merges with master except in the case of a release.
