\section{SOS Retrospective}

The sprint 4 retrospective had the same structure as that of sprint 3. There was much feedback, below is a filtered list containing only the most critical points. 

\begin{itemize}
    \item It is easiest to communicate with people if they work on site rather than from home.
    \item There should be established naming conventions for branches.
    \item \Glspl{skillGroup} were useful. It was nice with a dedicated group of people who knew how the server functions.
    \item \Glspl{fullStack} were useful.
    \item Make it clear how long a group will take to finish a user story.
    \item Report if the whole team goes on vacation.
    \item Always be available on Slack from 09 to 15.
    \item More backend transparency, like what version \gls{api} is used.
    \item One place where all process documents are found.
    \item The server group was too isolated from the rest of the project. 
    \item Next year \gls{PO} should create user stories in a story format instead of just a title. Sometimes it was hard to agree on how a story should be implemented. 
    \item \gls{POT} should only decide how the product should be, not how they are implemented.
    \item Reviews were used to enforce the reviewer's implementation thoughts, which should not happen. 
\end{itemize}

As this was the last sprint, the feedback was not used for changing the process, but more of an evaluation of the process and advice for next year.
