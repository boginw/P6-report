\section{SOS Retrospective}

The sprint 4 retrospective had the same structure as sprint 3. Everybody talked to a member of \gls{SMT}, and the feedback was gathered in a questionair. There were a lot of feedback this time, so this section only contains the most important points. 

\begin{itemize}
    \item It is easiest to communicate with people if they work on site rather than from home.
    \item There should be established naming conventions for branches.
    \item Meta groups were useful. It was nice with a dedicated group of people who knew how the server functions.
    \item Fullstack groups were useful.
    \item Make it clear how long a group will take to finish a user story.
    \item Report if the whole team goes on vacation.
    \item Always be available on Slack from 9-15.
    \item More backend transparency, like what version api is used.
    \item One place where all process documents are found.
    \item The server group was too isolated from the rest of the project. 
    \item Next year PO group should create stories in a story format instead of just a title. Some times it was hard to agree on how a story should be made. 
    \item PO should only decide how the product should be, not how things are implemented.
    \item Reviews were used to enforce reviewer's own implementation thoughts, this should not happen. 
\end{itemize}

As this was the last sprint, the feedback was not used for changing the process, but more of an evaluation of the process and advice for next year.
