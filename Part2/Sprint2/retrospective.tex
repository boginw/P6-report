\section{Retrospective} \label{s2Retrospective}

After the SOS retrospective \gls{T11} had our own retrospective. Here we reflected on sprint 2, and evaluated how the decissions we made in sprint 1 went\autoref{s1Retrospective}.

\subsection{Issues from previous sprint}

In sprint 1 we identified two key issues. The first was communication issues in the group, which led to some internal conflicts. We agreed that we should find alternative methods for resolving the conflicts. This could for example be to take a break for five minutes and find an alternative way to describe the discussion topic. We also found that we had a tendency to discuss things that was not necessary to discuss. We agreed to try to be better at listening to each other, and accept that some people have more experience in certain arias. 
\end{itemize}

We did not experience the first issue much during sprint 2 so we did not get to use the measures we agreed on in sprint 1. We think that two factors contributed to this. First we have been working together for a longer time and know eachother better. Therefore we can better navigate socially. We also felt that the discussion about the problem made us more aware and conscious about how we communicate.
Likewise, the discussion issue did not occur. We suspect the reason for this is that we do not have as many major decissions to make. This led to fewer opportunities to discuss.

\subsection{Issues this sprint}

We use a method where we merge text in to the report rather than writing it directly. All work have to be reviewed before it can go into the report. One of the problems we encountered was that we were too slow to review work that was finished. This resulted in some confusion about how much work had actually been made, since most work done was not actually in the report. After discussing it we agreed to see if realizing the problem and having the discussion was enough to solve the problem. If it kept being a problem we would assing a group member to be "review master" with the authority to make another group member do a review before doing anything else. 

Some members of our group had some issues with our \gls{PR} process was to strict. This demotivated them from making small changes, because the overhead of the process was to high. Other members liked the strictness of the process. To solve this we tried to make the process more flexible to satisfy both parities.

Another problem with our communication was that we had issues with getting into contact with each other when we worked from home. This yielded some issues when a group member needed information from another. The missing responds hindered the productivity of the group member. We agreed to handle this by establishing some procedures, e.g when we start work we should write it on the group communication channel.

We also found that we ended up having an uneven responsibility where some group members only worked on the user stories while others only worked on the report, effectivly splitting the group in two. We dealt with this in two steps. First we invert the responsibilities for next sprint so the group members got more equal responsibilities. After sprint with inverted responsibilities we wanted to have even responsibilities. We hoped that we would be more aware of the issue, but also that all members would be more comfortable with both the GIRAF project and the report.

The final issue was that some group members got asked by other groups to help with many tasks that were not part of our tasks. This meant that we could not plan our sprints to take this into account. Furthermore, some of the tasks we were asked to do were not a part of the tasks defined by the \gls{POT}. This meant that the tasks were not documented anywhere.

We agreed with \gls{POT} if other groups came with tasks to us we should redirect them to the \gls{POT}. They would then formalize the tasks, and assign it to a group.

\subsection{Things that went well}

Overall we are satisfied with the groups willingness to deal with internal problems. Members have not been afraid of voicing issues, doing so in an respectful and constructive manner. We like that we try to proactively solve our issues if we sense some tension between group members or if something in the process does not make sense. We are comfortable that issues are voiced and all members are willing to adapt our process. 