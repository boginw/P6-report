\section{Retrospective} \label{s2Retrospective}

In the end of sprint 2 we held a retrospective, to evaluate the how the process went, and to reflect on if managed to handle some of the issues from the process in sprint 1 see \autoref{s1Retrospective}.

\subsection{Issues from previous sprint}

Some of the issues from last sprint were:
\begin{itemize}
    \item communication issues in the group, which led to some internal conflicts. We agreed that we should find alternative methods for resolving the conflicts when they became an issue. This could for example be to take a break for five minutes and find an alternative way to describe the discussion topic. 
    \item We also identified the issue, that we had a tendency to discuss things that was not necessary to discuss. We agreed to try to be better at listening to each other, and accept that some people have more experience in certain arias. 
\end{itemize}

For the communication issues we did not experience it as much this sprint, but it was net necessary to use the measures we agreed on after sprint 1, we speculate that two factors mainly contributed to this, first we have been in a group for longer now, hence we know eachother better now, and can better navigate socially; secondly having the discussion about the problem made us more aware and conscious about how we communicate.
For the discussion issue, the problem also did not occur, we suspect the reason for this is that in sprint 2 we do not have  as many major decissions to make, so fewer opportunities to discuss occurred.

\subsection{Issues this sprint}

One of the problems we encountered were that when work had been done, we were too slow to review it so it could get in the report. This resulted in some confusion about how much work had actually been made, since most  work done was not actually in the report. After discussing it we first agreed to see if having a discussion and realizing the problem is enough to solve the problem, and then if the problem persist we will assing a group member to be "review master", with the authority to make another group member do a review before doing anything else. \\
Some members of our group had some issues with our \gls{PR} process was to strict, and demotivated them from making small changes, because the overhead of the process was to high, though other members liked the strictness of the process. To solve this we tried to relax the process in a more flexible way, so that we can satisfy both parities. \\
Another problem with communication occurred, we had issues with getting int to contact with eachother when working from home.  This naturally yielded some issues, because if one member needed information from another group member,  but that member is not responding, it hinders the productivity of the first member.\\
We agreed to handle it by encouragin communication from home by making some procedures, e.g when one starts working one should write it on the groups communication channel. We hope that actions like this, will kickstart the communication, and get the conversation going. \\
Another issue was that we ended up having an uneven responsibility split in the group, meaning that only some group members were writing code, while others only wrote documentation. We dealt with this in two steps, first we invert the responsibilities for next sprint, so that the group members get more equal responsibilities. Then after the sprint with inverted responsibilities we will have even responsibilities again. We hope that we will be more aware of the issue, but also that all members would be more comfortable with both writing code and documentation, hence ending up being more willing to take assignments from both areas work.\\
Then finally an issue we had were that some group members were asked by other groups to take some tasks that were not part of our groups sprint, meaning that we had not planned our sprints to take this into account. Further more some of the tasks we were asked to do were not a part of the tasks defined by the \gls{POT}, this meant that the tasks were not documented anywhere.\\
We agreed with \gls{POT} if other groups came with tasks to us, we should redirect them to the \gls{POT}, then they will handle formalizeing the task, and assigning it to a group.

\subsection{Things that went well}

Overall we are satisfied with the groups willingness to try and solve internal problems, members have not been afraid of voicing issues, and doing so in an respectable and constructive manner. We like that we try to proactively solve our issues, if we sense some tension between group members, or if something in the process does not make sense, it feels like it will be voiced and all members are willing to adapt our process. 