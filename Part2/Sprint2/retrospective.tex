\section{Retrospective} \label{s2Retrospective}

In the end of sprint 2 we held a retrospective, to evaluate the how the process went, and to reflect on if managed to handle some of the issues from the process in sprint 1 see \autoref{s1Retrospective}.

\subsection{Issues from previous sprint}

Some of the issues from last sprint were:
\begin{itemize}
    \item communication issues in the group, which led to some internal conflicts. We agreed that we should find alternative methods for resolving the conflicts when they became an issue. This could for example be to take a break for five minutes and find an alternative way to describe the discussion topic. 
    \item We also identified the issue, that we had a tendency to discuss things that was not necessary to discuss. We agreed to try to be better at listening to each other, and accept that some people have more experience in certain arias. 
\end{itemize}

For the communication issues we did not experience it as much this sprint, but it was net necessary to use the measures we agreed on after sprint 1, we speculate that two factors mainly contributed to this, first we have been in a group for longer now, hence we know eachother better now, and can better navigate socially; secondly having the discussion about the problem made us more aware and conscious about how we communicate.
For the discussion issue, the problem also did not occur, we suspect the reason for this is that in sprint 2 we do not have  as many major decissions to make, so fewer opportunities to discuss occurred.

\subsection{Issues this sprint}

One of the problems we encountered were that when work had been done, we were too slow to review it so it could get in the report. This resulted in some confusion about how much work had actually been made, since most  work done was not actually in the report. After discussing it we first agreed to see if having a discussion and realizing the problem is enough to solve the problem, and then if the problem persist we will assing a group member to be "review master", with the authority to make another group member do a review before doing anything else.
Another problem with communication occurred, we had issues with beeing available to eachother when working from home.  This yielded some issues, a
communication
documentation
work split

review master

Referat af mødet:
Kommunikations problemerne er blevet bedre, dog stadig et lille “problem” med at mikkel ikke rigtigt for snak hvad han har på hjertet.

- Problem med PR (Reviews) 
- Problem med kommunikationen over Slack. Vi skal bære bedre til at tjekke vores Slack for beskeder.
- Pull Request er trælse, fordi man ikke bare kan lave grammatiske ting uden at det er en pull request. Det tager for lang tid at komme igennem et review når alle grammatiske ting skal ind som suggestions.
- Det var rart at Git ikke blev alt for striks
- Mister overblik over hvor meget der bliver lavet, hvor langt vi er og hvad der mangler.
- Problem med at der er blevet arbejdet på ting som der ikke var kort på.
- Problem at 2 gruppe medlemmer stort set kun har lavet rapport skrivning og 2 medlemmer kun har programmeret.
- Vi har været gode til at snakke om tingene og bare få dem løst.
- Gruppen skal have ros for at man anderkender sine svagheder og man proaktivt gør ting for at “løse” sine udfordringer. Der bliver kommunikeret høfligt mellem gruppemedlemmerne, det modarbejder uvenskaber.
- Ting har været løst i “Ready for review”
- Måske man kunne implementerer en Review Master.
- Lave en Review guide kunne hjælpe med at gøre hele review delen nemmere.
- Der er forvirrende at gennemskue hvad målet er med projektet.
- Der skal være en tegl om at ingen arbejder med noget uden at man har snakket med de andre i gruppen.
