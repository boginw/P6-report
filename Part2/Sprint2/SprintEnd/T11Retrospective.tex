\section{T11 Retrospective} \label{s2Retrospective}

In sprint 1, we identified two key issues. The first was communication issues in the group, which led to some internal conflicts. The second issue was the tendency to discuss things that were not necessary to discuss.

We did not experience the communication issue during sprint 2, so we did not get to use the measures we agreed on in sprint 1. We think that two factors contributed to this. Firstly, we have been working together for a longer time and know each other better and can better navigate socially. Secondly, we also think that discussing the issue made us more aware and conscious of how we communicate.

Likewise, the discussion issue did not occur. We suspect the reason for this is that we do not have as many major decissions to make, which led to fewer opportunities to discuss.

\subsection{Issues From This Sprint}

Before merging any text into our report, we review the text. One of the problems we encountered this sprint was the inability to tell progress, as we were too slow to review the text, and thereby denying the text to be merged. After discussing it, we agreed to see if recognizing the problem and having the discussion was enough to solve the problem since this worked before. Otherwise, we would assign a group member the authority to make another group member review before doing anything else.

Some members of our group had issues with our \gls{PR} process, noting that it was too strict, which demotivated them from making small changes because the overhead of the process was too high. Other members liked the strictness of the process. To solve this, we tried to make the process more flexible to satisfy both parties.

Another problem with our communication was that we had issues with getting into contact with each other when we worked from home, which was a problem when a group member needed information from another. If group members asked questions, often, no answers were given, which hindered the productivity of the group. We agreed to handle this by establishing some procedures, e.g. before a member begins working, they should notify the group of their availability on the group communication channel, which encourages communication.

Counting up the hours spent by each member on writing in the report, and the hours spent developing for the \gls{giraf} project, we found a clear division between the group members. Some group members worked primarily on the report, while the others on the development, effectively splitting the group in two.

We dealt with this division in two steps. Firstly, we inverted the primary work area for each member for the next sprint, in an attempt to even out the number of hours in each boat. After the next sprint, we would remove the inversion and try to enforce each group member to work evenly in both the report and development. We hoped that we would be more aware of the issue, but also that all members would be more comfortable with working in both the \gls{giraf} project and the report.

The final issue was that some group members got asked by other groups to help with many tasks that were not part of our tasks, which made it difficult to plan our sprint, as it got disturbed by other tasks. Furthermore, some of the tasks we were asked to do were not a part of the tasks defined by the \gls{POT}, which meant that the tasks were not documented anywhere. We agreed with \gls{POT} if other groups came with tasks to us we should redirect them to the \gls{POT}. They would then formalize the tasks and assign them to groups.

\subsection{Things That Went Well}

Overall we are satisfied with the group's willingness to deal with internal problems. Members have not been afraid of voicing issues, doing so respectfully and constructively. We like that we try to proactively solve our issues if we sense some tension between group members or if something in the process does not make sense. We are comfortable that issues are voiced and all members are willing to adapt our process.