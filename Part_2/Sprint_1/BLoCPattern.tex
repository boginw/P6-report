\section{BLoC Pattern}

\subsection{Story behind}
BLoC stands for Business Logic Component and is a pattern that aims to isolate all
business logic away from UI code. The pattern was formulated by Paolo Soares and
Cong Hui, boths working on the Google AdWords platform. The story behind the birth
of the BLoC pattern is that at the Google AdWords team they have two applications the
web-application which is written in Dart-lang using AngularDart and a mobile-application
using Dart-lang and Flutter. The issue was that before Flutter existed they developed
their mobile-applications in native languages, meaning two code bases for mobile-application
and one code base for web-application. The maintaining of three code bases became an
almost impossible task and every change had to be repeated three times in three
different languages.

When Flutter was released Paolo Soares and Cong Hui saw the possibility to merge the
mobile-applications into one code base and thereby achieving only one language for the
development. They also initially thought that they would be able to reuse all the dart
written business logic from the web-application and thereby very rapidly develop the
mobile application. Trying to do so highlighted a very big issue in their web-application
which was that the business logic where mixed in with the UI logic.

[Show diagram of mixed business logic and UI logic]

So, if they wanted to re-use the business logic from the web-application they had to
first clean up the entiere application to clearly seperate business and UI logic. But
not only should it be seperated, but the business logic should also be platform independent
meaning that it should be useable with out any modifications on both the web-application
and the mobile-application. The pattern they ended up with as a solution for this was
named BLoC (Business Logic Component).

[Show diagram of seperated business logic and UI logic]

The pattern alone does not ensure that business and UI logic are not mixed, so along
side the pattern they conformed a list of non-negotianal rules that is to be upholded
by all developers, no exceptions.

\subsection{What is a BLoC}
A BLoC i simply a class that utilizes streams for communication, this is because
not all platforms are reactive in nature, Flutter is a good example of this, while
AngularDart supports two-way databinding out of the box, Flutter needs to either specificly
do a entire state update or listning to a stream to get changes reactively. Therefore
a BLoC must be implemented such that all inputs are sinks and outputs are streams.

