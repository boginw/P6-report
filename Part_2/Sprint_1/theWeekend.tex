\section{The Weekend}
When moving to Flutter, the Frontend Group put our team in charge of the move. 
There was two main issues associated with moving the entire frontend from one framework to another mid semester.
The first issue was, that many user stories involved changing the frontend. Therefore many teams were not able to work while the change was happening.
The second issue was, that everybody needed to be involved in the move. 

The way we handled the first issue, was to make the move as quick as possible, so other Teams could continue working on the app. 
To do this, we worked through the weekend, and got as much done as possible in the time available. The next week we presented our work at a Flutter meeting for all the groups. 
During the weekend, we got help from a couple of members from other teams. 
When we made the move, we had two points of focus. In the old application, the code that connected to the API, here called the \gls{fapi}, was auto generated by the software framework Swagger.
This resulted in several issues with communicating with the API, and made us unable to add further functionality to the \gls{fapi}. 
The user interface of the application and the \gls{fapi} were developed in parallel, where some members worked on one, and others on the other. 
The result of the weekend was, that some basic rules and guidelines for developing the user interface and the way to add functionality was defined, some basic \glspl{screen} were added, and an almost functional \gls{fapi} with belonging models were done. 
To make sure the whole GIRAF year were included, and resolve the second issue, we did not try to make the UI or functionality complete, but rather make it possible for everybody to work on the project together without the project being chaotic, by issuing guidelines. 

Before the Flutter meeting, we worked together with the PO group, to make some user stories. 
On the meeting we presented our work, described in short what Flutter is, and made a demonstration of how to make a \gls{screen} with functionality and communication with the API.
The user stories were then given to the different groups, and they begun working on them. 