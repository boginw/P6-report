\chapter{Sprint 1}


The \gls{PO} had been in contact with the customers, and they expressed that the most important thing for them is to have a better functioning Weekplanner.
There are shortcomings functionality wise  in the current application, and lack of stability, there are issues with the user interface being confusing and in some cases inconsistent. The customer also wants offline functionality, primarily for two reasons, they do not have a stable enough wireless coverage, and the are often on trips to places without internet.

\section{The original purpose}
% What the sprint should have been
%   Defined purpose
% Our original job
%   Short about our stories
Before the first sprint the \gls{PO} had been in contact with the customers, and they expressed that the most important thing for them is to have a better functioning Weekplanner.
There are shortcomings functionality wise  in the current application, and lack of stability, there are issues with the user interface being confusing and in some cases inconsistent. The customer also wants offline functionality, primarily for two reasons, they do not have a stable enough wireless coverage, and the are often on trips to places without internet.
\\
The sprint started with a joint sprint planning with all the development groups present. The \gls{PO} group chose to update the Weekplanner application as the sprint focus, based on the information from the customers. Then the \gls{PO} presented the user-stories in prioritized order, then gave the groups time to discuss the different stories internally. After discussing the stories the groups then had three votes, all the groups then voted on the stories they wanted, and if there were too many groups requesting the same user story, then some negotiation was needed in order to make a fair allocation of assignments. The voting was intended as a way for all the groups to get a fair chance of getting the assignments they wanted, instead of first come first served approach.\newline

\section{Our userstories}
We took three tasks this sprint.
\begin{description}
    \item [\#14] As a citizen I would like the icons to be consistent throughout the system so that I instinctively know their meaning.
    \item [\#15] As a citizen I would like to be able to choose how many days I see at a time on my weekplanner, so that it fits my personal preference.
    \item [\#19] As a user I would like the icons to be updated so that they are modern and easy to understand.
\end{description}

\subsection{Updating the icons}
Issue \#14 and \#19 are very closely related and can be solved in tandem. Issue \#14 is that some of the icons in the application are unclear, and have multiple meanings depending on where in the application they are used. The users have expressed concern regarding the simplicity and intuitiveness of the application and would prefer if the application was intuitive enough so that training is unnecessary.
As a part of this task all icons and their use will have to be mapped, to see which icons make sense, and as far as possible only have one meaning. As for issue \#19 it makes sense to do simultaneously, as while we map the icons and their meanings, updating them alongside reduces some redundant work. 

The \gls{PO} group made a design guide in cooperation with the customers, the assignment will then be to make the weekplanner application respect this design guide.

\section{Changing how many days the Weekplanner shows}\label{sec:weekPlannerDaysToShow}

Issue \#15 expresses a user need for being able to change how many days the Weekplanner shows at a time. This ability is a need expressed by some citizen, as they can feel a bit overwhelmed by all the tasks, so they would like to be able to show fewer days, or even just one.
Initially, we thought this would be a trivial task because we believed it would be some minor \gls{XAML} changes, but after looking at the code and discussing the assignment with the \gls{PO} group, it turned out that this was not the case. We discovered that the way the current weekplans are stored, are in separate, unconnected plans, meaning that a weekplan has no connection to another weekplan. This non-connection means that if the Weekplanner has to show data from two weeks, it has no way of knowing which weekplan comes after the current.
To solve this we have to find an alternative way of storing the plans, possible in a more calendar inspired way.



The sprint started out with a collective sprint planning with all the development groups  present. The \gls{PO} group chose updating the Weekplanner application as the sprint focus, based on the information from the customers. Then the \gls{PO} presented the userstories in prioritized order, then gave the groups time to discuss the different stories internally. After discussing to stories the groups then had three votes, all the groups then voted on the stories they wanted, and if there were too many groups requesting the same userstory then some negotiation will be needed in order to make a fair allocation of assignments. The voting was intended as a way for all the groups to get a fair chance of getting the assignemnts they wanted, instead of first come first served approach.\newline

\section{Our userstories}
We took three tasks this sprint.
\begin{description}
    \item [\#14] As a citizen I would like the icons to be consistent throughout the system so that I instinctively know their meaning.
    \item [\#15] As a citizen I would like to be able to choose how many days I see at a time on my weekplanner, so that it fits my personal preference.
    \item [\#19] As a user I would like the icons to be updated so that they are modern and easy to understand.
\end{description}

\subsection{Updating the icons}
Issue \#14 and \#19 are very closely related, and can be solved in tandem. Issue \#14 is that some of the icons in the application are unclear, and have multiple meanings depending on where in the application they are used. The users have expressed a concern regarding the simplicity and intuitiveness of the application, and would prefer if the application was intuitive enough so that training is unnecessary.
As a part of this task all icons and their use will have to be mapped, to see which icons make sense, and as far as possible only have one meaning. As for issue \#19 it makes sense to do simultaneously, as while we map the icons and their meanings, updating them alongside reduces some redundant work. 

The \gls{PO} group made a design guide in cooperation with the customers, the assignment will then be to make the weekplanner application respect this design guide.

\section{Changing how many days the Weekplanner shows}\label{sec:weekPlannerDaysToShow}

Issue \#15 expresses a user need for being able to change how many days the Weekplanner shows at a time. This is a need expressed by some citizen, they can feel a bit overwhelmed by all the tasks, so they would like to be able to show fewer days, or even just one.
Initially we thought this would be a trivial task, because we believed it would be some minor \gls{XAML} changes, but after looking at the code and discussing the assignment with the \gls{PO} group it turned out that this was not the case. We discovered that the way the current weekplans are stored, are in separate unconnected plans, meaning that a weekplan has no connection to another weekplan. This means that the Weekplanner has to show data from to weeks, it has no way of knowing which weekplan comes after the current.
To solve this we have to find an alternative way of storing the plans, possible in a more calendar like way.

\section{Moving to flutter}
% Why did we suddently move to flutter?
% What is flutter? 
% How did we move to flutter?
% Our role in moving to flutter
% What was the status of the flutter project after the weekend?
%   Non autogenerated API, why
%   Bloc stucture 
% Workshop and koordination 
%
\section{SOS Sprint Review}
In the end of sprint 1 \gls{G19} held a review meeting. This section describes that meeting.
The \gls{G19} \glspl{devTeam} started with features for the Xamarin implemented app. However, the Xamarin ecosystem was disallowing a lot of the developers to work since the Xamarin project has abandoned Linux support. Since there were roughly 40\% of the developers on the Giraf project running Linux, this was a significant issue.

Due to the Linux issue of Xamarin, the front-end meta group decided that a move to Flutter was necessary. At the end of Sprint 1 the transition from Xamarin to Flutter where not at a release ready state. The status of the new Flutter app was that the architecture, data models, API module and some initial demo-screens where implemented.

Within Sprint 1 the first alpha version of the Flutter app core where released to the \glspl{devTeam} and the \gls{PO} group had prepared new user stories for the \glspl{devTeam} to tackle. The developers behind the alpha core held a release presentation where all developers of the \gls{G19} where present. This presentation was an introduction to Flutter, the architecture of the new core and a coding session where a user story where implemented. Afterward, the remaining user stories were delegated to each \gls{devTeam}, leaving roughly one week before the end of the sprint. The short timeline ment that most \gls{devTeam} were not  able to finish their story before the end of the sprint. Never the less the overall feedback was that the developers were happy with the transition to Flutter, they felt that the development where more rapid and the UI was more comfortable to implement. Most \glspl{devTeam} nearly finished with their stories, but most of them were blocked by lacks in the core mainly: The ability to transmit data from one component to another, The ability to implement widget tests, and a bug in the authentication system.

The \gls{PO} group where satisfied with the progression doing this sprint, they considered the move to Flutter an elimination of serious technical depth and thereby an improvement to the quality of the product.

\section{SOS retrospective}
%Hvad was decided on the meeting?
In the end of the sprint, all the groups were gathered to carry out the SOS sprint retrospective. 
At the meeting, we formed subgroups that consisted of at most one member from every team. 
The subgroups then had fifteen minutes to find ideas for changes to the process. 
Thereafter the ideas were reviewed, and duplicates removed. Everybody could then vote for three ideas. 
After this, the process group took the feedback, and made changes accordingly. 
Most people were happy with the overall process, and the way the groups communicated. 
The change to Flutter, however, made it hard to decide if the process structure was successful, because it became chaotic.
The release was also cancelled, because the Flutter application was not ready to launch. 
One change made to the process was, that stand up meetings should be on days with lectures, because some groups found it annoying that they needed to come to the university for just one meeting.
This were on request from several members of G19. 
The process group also changed the way Sprint planning is carried out.
Rather than giving the groups all the tasks estimated for the sprint up front, they now get to take one from the backlog when needed. 
The sprint planning is then more of an sprint intro meeting where the groups are informed of the purpose of the sprint. 

\section{G11 Retrospective}
%What was good about our process
%What was bad about our process
%Changes to the process
