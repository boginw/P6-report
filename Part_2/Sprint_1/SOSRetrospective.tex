\section{SOS retrospective}
%Hvad was decided on the meeting?
In the end of the sprint, all the groups were gathered to carry out the SOS sprint retrospective. 
At the meeting, we formed subgroups that consisted of at most one member from every team. 
The subgroups then had fifteen minutes to find ideas for changes to the process. 
Thereafter the ideas were reviewed, and duplicates removed. Everybody could then vote for three ideas, which they found most importaint. 
After this, the process group took the feedback from the vote, and made changes accordingly. 
Most people were happy with the overall process, and the way the groups communicated. 
The change to Flutter, however, made it hard to decide if the process structure was successful, because the process flow got broken during the hasty move to Flutter. Therefore, many practises were never really used, and there were nothing finished to release.
The sprint's release was also cancelled, because the Flutter application was not ready to launch. 
One change made to the process was, that stand up meetings should be on days with lectures, because some groups worked from home generally, so it would suit them better if the meetings were placed close to lectures, where they would already be pressent.
This were on request from several members of G19. 
The process group also changed the way Sprint planning is carried out.
Rather than giving the groups all the tasks estimated for the sprint up front, they now get to take one from the backlog when needed. 
The sprint planning is then more of an sprint intro meeting where the groups are informed of the purpose of the sprint. 
