\subsection{Moving to Flutter}
We were put in charge of the actual Flutter move. 
There was two main issues associated with moving the entire frontend from one framework to another mid semester.
The first issue was, that many user stories involved changing the frontend. Therefore many teams were not able to work while the change was happening.
The second issue was, that everybody needed to be involved in the move. 

The way we handled the first issue, was to make the move quickly so other teams were blocked as little as possible.
To do this our team, as well as a few members of other teams, worked through the weekend, to get the basics of the app ready. After the weekend we presented the work on a workshop for everybody in the project, and gave further tasks to other teams.

In the Xamarin application, we had a lot of trouble with the code that communicated with the API, here called \gls{fapi}. This was because it was autogenerated by an outdated version of a framework called Swagger. We found out that it was not possible to add further communication between the UI and the API. Therefore, we wrote the code ourselves when we moved the project to Flutter. The UI elements and the \gls{fapi} was developed in parallel. Some people worked on the UI, and some worked on the \gls{fapi}

At the time of the workshop, we had a lot of the application finished. The \gls{fapi} part was almost completely done, all the data models were made, there were unit tests on everything, we had basic rules for developing the UI, and added some of the screens. 
At the workshop, we gave a presentation about Flutter, how you use it, the work that was finished and a live coding of one of the screens. This way everybody had a feeling of what Flutter is, and the work that was done. 
After the presentation, each group got a userstory that involved the UI. We had developed the userstories with the \gls{PO} group so it would be easy to coordinate and complete the rest of the move to Flutter.