\section{Server Move}

Since 2016\cite{SW611F16}, the Giraf project used self-hosted GitLab for their Git solution , and Gogs before that\cite{SW603F15}. The GitLab servers crashed on numerous occasions, even though it was not short of resources, having 2.5 GHz cores and 4GB RAM. Rather than deliberating how to stabilize the servers, \gls{G19} decided to move the GIRAF project to a hosed solution. This section describes how we migrated the Giraf repositories and \gls{ci} to GitHub and Azure.

We chose GitHub as the only candidate, since GitHub provides free private repositories to students and most software students are familiar with their services. Therefore we concluded that this would a good choice for the project. It should also be noted, that this has been done before, in 2013\cite{SW601F13}, but in 2014, they were using Gitolite\cite{SW613F14}, the reason for this switch is unknown.

\subsection{Method}

To perform the actual migration we used the three commands, shown in \autoref{lst:gitMigrate}. The first command tracks all branches in the current repository, the second adds a new remote to the repository called `github,` and the third command pushes everything to GitHub.

\begin{lstlisting}[language=bash,label={lst:gitMigrate},caption={Git Migration code}]
for remote in `git branch -r`; do git branch --track $remote; done
git remote add github $1
git push --mirror github
\end{lstlisting}

\section{Continous Integration}

GitLab also provided \gls{ci} for the different repositories, and with the switch from GitLab, we needed a suitable replacement.

While there are plenty of \gls{ci} services available, most of them are not free, and we could not wait for funding by the university since the \gls{ci} is an integral part of the workflow. Therefore we limited our seach to free options.  We recommend reconsidering this decision at a later time, as even though there are plenty of \gls{ci} services that are free for Open-Source projects, some have restrictions when not paying.

Furthermore, we needed the ability to deliver iOS applications, which requires MacOS, so our \gls{ci} service must support running MacOS. 

Lastly, we needed a \gls{ci} that had plenty of build minutes, and concurrent builds, as there was expected many commits each day on the different repositories.

On \autoref{tbl:ci_comparison} we list the different \gls{ci} services that were considered in the search. 

\noindent\begin{longtable}[]{@{}lrrrl@{}}
    \caption{\gls{ci} comparisos}
    \label{tbl:ci_comparison}\\
    \toprule
    Service & Price & ~Build Minutes & Concurrency & ~Mac\tabularnewline
    \midrule
    \endhead
    GitLab & 19\$/user & 10,000 & NA & ~No\tabularnewline
    CircleCI & 0 & 1,000 & 1 & ~Yes, for \$129/mo\tabularnewline
    AppVeyor & 0 & ~ NA & NA & No\tabularnewline
    TracisCI & ~ 0 & ~ NA & ~ NA & ~No ~\tabularnewline
    Azure & 0 & ~ Unlimited & ~10 & Yes\tabularnewline
    \bottomrule
\end{longtable}

We ended up choosing Azure, because it is the option that best fit our criteria. 