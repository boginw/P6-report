\section{Final process}
We defined a process at the beginning of the semester. This process evolved through the development period. This section describes the final process. 

The \gls{SOS} process started with seven activities (see \ref{sec:crossTeam}). The sprint planning and the sprint review were removed from the process as they did not fit us. 

The \gls{SOSStandUp} activity ended up being held two times a week, still with the same format as initially. 

The retrospective changed. In the beginning, we voted on the change proposals gathered during the process discussion, but it ended up as a questionnaire where everybody answered if they agreed, did not care or disagreed, which gave more nuance.

The \gls{skillGroup} meetings were not held once a week, but instead when needed because we found it unnecessary if there was not anything new. 

The \gls{PR} process remained the same throughout the semester. We decided that people should put more effort into using Slack so it was possible to get into contact with each other. 

The \gls{T11} process changed a lot. We did have a sustainable pace most of the time and ended up in a position where the members did not have too much work, but we did not need to rush in the end either. The test-driven development was not practiced, but the pair programming principle was used sometimes. We did not have daily standup meetings as we forgot those. 

We kept the retrospective activity and introduced a sprint planning where we tried to estimate how much work we could get done. The retrospective was used to work on our internal conflicts. 
