\section{Final process}
We defined a process in the beginning for the semester. This process evolved through the development period. This section describes the final process. 

The \gls{SOS} process started with seven activities(see \ref{sec:crossTeam}). The sprint planning and the sprint review were removed from the process as they did not fit into the process. 

The \gls{SOSStandUp} activity ended up being held two times a week, but ended in the same format as initially planned. 

The retrospect changed. In the beginning we voted on the change proposals gathered during the process discussion, but it ended up as a questionair where everybody should answer if they agreed, did not care or disagreed. This gave more nuance, as it let people respond to every item and made it possible to show that they disagreed.

The skill group meetings was not held one time a week, but instead when needed. This was because we found it unnecessary when there were not anything important to talk about. 

We ended up with mostly the first defined practices, as we followed the GIRAF adaption of GitFlow, used the issue report protocol to when we found issues and so on. 

The PR process remained the same through the semester, although we put more focus on the speed of the review, so it did not take too long. It was also decided that people should look on Slack more frequently so it was possible to get into contact. 

The \gls{T11} process changed a lot. We did have a sustainable pace most of the time, and ended up in a position where the members did not have too much work, but we did not need to rush in the end either. The test driven development was not followed and mostly forgotten, but the pair programming principle was used sometimes. We did not have daily standup meetings as we forgot those. 

We kept the retrospective activity and introduced a sprint planning where we tried to estimate how much work we could get done. The retrospective was used to work on our internal conflicts. 