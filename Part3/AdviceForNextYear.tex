\section{Advice for next year students}

We have learned a lot of new things during this semester, both about working a multi project and on the GIRAF project. This section contains our advice for future developers on the GIRAF project. This is based on both good and bad experiences. 

\subsection{Project communication}
We used Slack to communicate with other development teams during the semester. Slack has the advantage of multiple channels and makes it possible to communicate with a specific group of people by joining their channel. One problem we experienced with communication was that people often answered questions in messages rather than in the message thread. The result was a messy channel and it could be hard to find out what belonged to which messages. It was much easier to find information when the threads were used. We recommend that a practice where a question is answered in the thread. 

Another problem was that people used slack to send documents, like summaries, rather than a shared folder. This made it difficult to find a document and get an overview of the available documents. We strongly recommend a shared folder, to store all documents so every group has access. 

\subsection{Structure of the teams}
We were happy with the use of full stack teams, because it made teams work independent of each other, and reduced the dependencies and waiting a lot. We believe that less dependencies means less conflicts, because there are fewer reasons for conflict. It also makes all groups familiar with the system, and we found it easy to ask other groups for help or a different perspective. The meta groups made sure that there were people who were specialized in each part of the system, and that every group were involved in big decisions. We recommend to use full stack teams. 

The product owner team and the process team was helpful, but excluded all other groups from the decisions. The project owner role is also a lot of work for one team, though it gives a good overview of the product. We recommend to consider making the teams into meta groups, so the work and responsibility is distributed over all teams, and every team is a part of the whole process, and get to communicate with the costumers. Alternatively, the project owner team should act as a project owner and not a project manager, and do the sprint planning with other teams and not just by themselves. We see no reason for a process group rather than a meta group, as all teams should take part in the process decisions. 

\subsection{???}
We chose to move the frontend even though last year moved it and advised against moving it. We also moved a few other things. We found that we had to move the application because we could not get it to work, but we advise against moving things if it is not necessary. Next year students will most likely move it again anyway. It is often tempting to start from scratch because we think we can do better, but remember to weigh the cost against the benefit. In regards to Flutter, it may look confusing in the beginning, but once you get used to it, it is very nice to work with and easy to use.

When any work was merged into development, it first had to be reviewed by other groups. This was both good for the project, as we ensured some code standard, and as a developer because we got new eyes on the work. Another benefit is that we got experience with other peoples code and coding styles. We strongly recommend external reviews of all code.
