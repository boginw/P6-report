\section{Advice For Next Year Students}

This section contains our advice for future developers on the \gls{giraf} project. This is based on both good and bad experiences.

\subsection{Project Communication}

We used Slack to communicate with other development teams during the semester. Slack has the advantage of multiple channels and makes it possible to communicate with a specific group of people by joining their channel. One problem we experienced with communication was that people often answered questions in new messages rather than in the thread belonging to the message containing the questions. The result was a messy channel and it could be hard to follow the discussions. It was much easier to find information when the threads were used. We recommend a practice where a question is answered in the thread.

Another problem was that people used slack to send documents and assets, rather than a shared folder. This made it difficult to find a document since no overview of the available documents existed. We strongly recommend a shared folder, to store all documents so every group has access.

\subsection{Structure of The Teams}

We were happy with the use of full stack teams because it made teams work independently of each other. We believe that with more independency fewer conflicts occur since there are fewer elements to conflict. It also makes all groups familiar with the system, and we found it easy to ask other groups for help or a different perspective. The meta groups made sure that there were people who were specialized in each part of the system, and that each group was represented when debating important decisions. We recommend using full stack teams.

The product owner team and the process team was helpful but excluded all other groups from participating in many of the decisions. We recommend to consider making the teams into meta-groups, so the work and responsibility are distributed over all teams, and every team is a part of the whole process, and get to communicate with the customers.

\subsection{Project Work}

Even though the students from the year before us, advised against rebuilding the Weekplanner. We had to rebuild it since the old weekplanner application was not able to compile on certain operating systems. But we would strongly advise against doing the same next year. It can often be tempting to rebuild an old codebase, but be aware of the costs and benefits of doing so. In regards to Flutter, it may look confusing in the beginning, but once you get used to it, it is very nice to work with and easy to use.

When any work was merged into development, it first had to be reviewed by other groups. This was both good for the project, as we ensured some coding standard, and as a developer, because it encourages knowledge sharing. Another benefit is that we got to experience with reading other peoples code and coding styles. We strongly recommend external reviews of all code.