\section{Conclusion}
The goal of this report was to learn how to work in a larger work environment. This section will conclude on what we learned about teamwork and large projects. 

When we got the project we had to define a suitable process. We defined a Scrum like process, called \gls{SOS}, with four sprints and all the Scrum activities where one group acted as a developer. All groups worked across the system, rather than on a specific part of the system, and were involved in any major decisions. 

During the semester there were an incident where one group waited for another to finish their work in the backend. This led to much frustration as the other group did not prioritize this work and the first group had to wait. This is a good example of the benefits of working across the product rather than groups having to wait for each other often. Another benefit of working across the project is that all developers get an idea of how the system work.
At the end of each sprint we evaluated and adapted the process. The changes was mostly changes to the practices like how to handle a blocked user story or how to communicate to each other.

Our development team, \gls{T11}, had our own process that ran parallel with the \gls{SOS} process. This was possible because the \gls{SOS} process did not define how a group should do their work, only what it should include, when it should be done and what meetings members should attend. 

In both the \gls{SOS} and \gls{T11} process, the one of the problems was communication. This could both be misunderstandings and poor coordination of the work. During the project we improved our communication, which led to fewer conflicts and better teamwork. Many problems were reviled in the retrospects, and these meetings proved vital to the process. 

Most of the groups were present on the university every day, but one group always worked at home. This made it hard to communicate with them, and their work were often late or lacking. We found it useful to be at the same location as the other groups, because it made it easy to get help or feedback.

All work was done on an application called the Weekplanner or the backend of the system. There were a version of the Weekplanner when we started. We tried to work on that version but concluded that it was too poorly documented and structured, and the framework not working. We made a new version of the application in another framework, thereby removing the old work. We also moved all repositories from Gitlab, where they were originally hosted, to Github. Last years students also moved a many things and made the Weekplanner from scratch. This could imply the difficulty of accepting already made choices in development, instead of accepting them. There were however too many problems with the existing Weekplanner for us to continue work on it, and the costumers liked the new application better than the old one.

All work was done using the Git Workflow branching, and no new content could be added to the development branch without a pull requests. The pull requests were review by other groups, so it upheld the standards set for the project. The requirements included both testing and proper descriptions of the code. This procedure made sure the product was more stable and well documented. The documentation is also necessary when next year students continue working on the product. 

Working on a large project both have costs and benefits. It can be difficult to coordinate many developers, but if a good structure is provided there is a lot of resources available. We found that it was good to be able to define our \gls{T11} process as well, because we could make one that fit us rather than one process fits all. 