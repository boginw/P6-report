\section{Conclusion}

The goal of this report was to learn how to work in a larger work environment. This section will conclude on what we learned about teamwork and large projects. 

When we got the project, we had to define a suitable process. We let us inspire by the \gls{SOS} process. All groups worked in \glspl{fullStack}. 

During the semester there was an incident where one group waited for another to finish their user story in the backend, which led to much frustration as the other group did not prioritize that user story. This incident is an excellent example of the benefits of working in \glspl{fullStack}, as long as the \gls{POT} defines user stories in complete vertical slices. Another benefit of working in \gls{fullStack} is that all developers get insights into all segments of the system.

At the end of each sprint, we evaluated and adapted the process. The changes were mostly changes to the practices like how to handle a blocked user story or how to communicate with each other.

Our development team, \gls{T11}, had a work process that ran parallel with the \gls{SOS} process, which was possible because the \gls{SOS} process did not define how a \gls{devTeam} should work.

In both the \gls{SOS}'s and \gls{T11}'s process, one of the problems was communication, which could be misunderstandings or poor coordination of the work. During the project, we improved communication in both \gls{G19} and \gls{T11}, which led to fewer conflicts and better teamwork. Many problems were identified in the retrospects, and these meetings proved vital to the process. 

Most of the groups were present at the university every day, but one group tended to work from home, which made it hard to communicate with them and complicated the workflow. We found it useful to be at the same location as the other groups because it made collaboration easier.

We tried to work on the old Xamarin application but concluded that with the number of problems with compiling on certain operating systems and our difficulties in understanding the documentation, we decided to build the application anew. 

Other groups reviewed pull requests to ensure the standards for the project. The requirements included both testing and proper descriptions of the code. This procedure helped the developers with sharing tacit knowledge and mitigated blatant errors.

During the semester we learned a lot about working in a large project. When working in small teams it is easier to communicate and coordinate, which can be difficult in a larger project consisting of multiple teams, but it is vital. 

The ability to define our \gls{T11} process, was essential in our participation during the semester, since we could make changes internally and thereby reaching a work process that fit us rather than one process fits all. 

With the \gls{giraf} project, we have learned to communicate between and within groups, organize tasks and identifying when there is a need for a change in the work process. 
