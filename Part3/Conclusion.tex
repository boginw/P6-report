\section{Conclusion}
The goal of this report was to learn how to work together in teams and how coordinate work between several teams. This section concludes on what we learned about teamwork and group coordination. 

The teams previous years worked on specific parts of the system. This year we chose to work fullstack instead, where all teams work across the whole system. We followed a Scrum process with four sprints, with one group, \gls{SMT}, responsible for defining and changing the process, and another responsible for communication with costumers and defining work tasks, \gls{POT}.

We had three meta groups that were responsible for each their part of the system, frontend, backend and server. There were one member of each team in each group. They made decisions about their part of the system, and helped their groups with work on it. 

In the end of each sprint, we evaluated the process and \gls{SMT} adapted it accordingly. Most of the changes were small, like when to hold a meeting, but we removed two activities from the process, review and planning, because people found it unnecessary. 

The meta groups worked well because all groups got to be a part of the decisions without too many people making the choices. There were some issues with \gls{POT} and \gls{SMT}. Both teams made decisions without the involvement of all the other groups, though \gls{SMT} made them after a retrospect meeting with all members of \gls{G19}. It was also difficult to follow what choices had been made.

There were good and bad things about a \gls{POT}. They were able to make quick decisions and had a good overview of the situation, but they made all the decisions without the involvement of the other groups. The reason the sprint planning was removed from the process was that tasks were divided but already defined, instead of the developers negotiating with a PO. 

Some issues could be solved by having PO and Process meta groups, so all groups could be part of the decision making. This could also make a more equal division of responsibilities rather than two groups having to do all the work.

In the \gls{T11} we followed the same sprints as \gls{G19}, with a planning and a retrospect activity. We focused a lot on communication and issue handling, and found that good communication and respect is key to a successful collaboration. 

In the beginning we had some issues with miscommunication and discussions where the parts misunderstood each other. After we worked on this with more conflict handling, our teamwork improved. 
