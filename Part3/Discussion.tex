\section{Discussion}

When we worked on the report, we encountered some points worth discussing. This section discusses the points we find most interesting. 

\subsection{Realizing the product vision}
The \gls{POT} defined a product vision at the beginning of the project, which consisted of four points:

\begin{enumerate}
    \item Make the Weekplanner more useful
    \item iOS compatibility
    \item Offline accessibility
    \item An easier handover to next year
\end{enumerate}

The first point focused on two aspects, user experience and performance. The \gls{POT} met with the customers at the end of the semester, and they were thrilled with the new Weekplanner. They also mentioned that it was intuitive to use contrary to the old application. The database enhancement made the Weekplanner perform significantly better. There are still ways to improve the Weekplanner and features missing, but all in all, it has improved.

The use of Flutter as the programming framework makes the application compatible with both Android and iOS devices per design. The only problem with iOS compatibility we found is, that you always need a computer running macOS to compile to the iOS devices, which meant only a few developers could test the application on iOS. This restriction is independent of the what framework used.

We decided not to focus on offline availability as we could not find time to do so.

A smoother handover to next year's students was a key goal for the students this year as we found the start of the project difficult because things were either poorly or not at all documented, and most available information was hidden in an old report. We put much work into the Wiki repository and hope that next year students find it useful.

We did not completely fulfill the original goals, but given that we made a new Weekplanner application, we believe that the Weekplanner is a better product after this semester. We also believe we have accommodated the handover for next years, students better than what the previous year did.

\subsection{The project start}

When we got the project handed over, we had a hard time understanding what the project was and what to do. We were also told to read the old reports and create our overview rather than being guided from the start, which came with both benefits and disadvantages.

A benefit from the lack of initial guidance was that we got the opportunity to learn about coordination in a large scale environment. A disadvantage is that we have to heavily rely on last years students or any potential experience students might have with project management. Our process also meant that it was only the \gls{ST} that got any experience with coordination. 

We suspect that the project handover process plays a vital role in how the whole semester will turn out. If the handover is messy, it can be harder to understand the structure and direction of the received project, which in turn can result in an unwillingness to adapt to the system received. 

If students are unwilling to adopt the previous project, then they might be likely to rewrite entire segments of the system, and one can imagine this leading to the \gls{giraf} project as a whole going backward. We think it might be beneficial to attempt to improve the transition phase either by having more cooperation with the previous year's developers or by having more help from the university at the start.

\subsection{POT and PT}

Many decisions in the project were made by \gls{POT} and \gls{ST}, which made communication between the decision makers easy and fast but excluded all other members of \gls{G19} from the decision process. We also experienced problems communicating which decisions and changes were made.

In the activity sprint planning from Scrum, developers, and \gls{PO} negotiate the user stories included in the sprint. When the sprint backlog is defined, the developers choose how to implement the stories themselves. In \gls{G19}, the \gls{POT} seemed to act a bit like project managers rather than their assigned role. It also felt like they micromanaged the delegation of tasks a bit too much as developers had to ask them for permission before taking a story. They did this in an attempt to mitigate merge conflicts and blocking, but resulted in a significant overhead, especially for small stories. The micromanagement could be avoided with better user stories and a better overview of the tasks.

After having tried the \glspl{skillGroup} for the frontend, backend, and server we think that it might be a good idea if the \gls{POT} and \gls{ST} roles also were implemented as \glspl{skillGroup} so that all groups have a more direct say in the process and tasks.
