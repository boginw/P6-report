\section{Discussion}

When we worked on the report we encountered some points worth discussing. This section discusses the points we find most interesting. 

\subsection{Realizing the product vision}
The \gls{POT} defined a product vision in the beginning of the project. This consisted of four point:
\begin{itemize}
    \item Make the weekplanner more useful
    \item iOS compatibility
    \item Offline acceptability
    \item Easier handover to next year
\end{itemize}
The first point focused on two things, user experience and performance. The \gls{POT} met with the costumers in the end of the semester, and they were very happy with the new weekplanner. They also mentioned that it was intuitive to use contrary before. The enhancement done on the database performance and smaller amount of pictures per request have made the weekplanner better performing. There are still ways to improve the weekplanner and features missing, but all in all it have improved.

The user of Flutter as the programming framework makes the application automatically compatible with both Android and iOS devices. The only problems with iOS compatibility is, that you need a Mac computer to compile to the iOS devices, so only a few developers can test the features on iOS. 

We decided not to focus on offline availability as we could not find time to do so.

Easier handover to next year was a key goal for the students this year as we found the start of the project difficult because things was either poorly or not at all documented, and most available information was hidden in an old report. We put a lot of work into the Wiki pages, and hope that next year students find it easier.

We did not fulfill the original goals, but given that we made a new weekplanner app from scratch, we feel like we have achieved a lot and can handover a better product than we got.

\subsection{The project start}
When we got the project handed over, we had a hard time understanding what the project was and what to do. The application was so poorly documented and un intuitive that it took a long time to find out what was what.

We were also told to read the old report and make a structure on our own rather than being told what to do. This came with both benefits and disadvantages.

The good thing about this was that we got to learn to coordinate ourselves. This adds to the learning process and gives us some experience. The bad is that we have to heavily rely on last years students or be lucky that some of our students know what to do. The structure we had also mend that it was only the \gls{POT} that actually got any experience with coordination.

One way to improve this could be to get somebody with experience in this type of project to consult on both the beginning and during the project. This would both make it possible to gain coordination experience and learn how projects are led in the industry. The way the handover was this year could make a good or a bad process, and a bad process is in the way of learning. 

\subsection{\gls{POT} and \gls{SMT}}
Many decisions of in the project were made by \gls{POT} and \gls{SMT}. This made communication between the decision makers easy and fast, but excluded all other members of \gls{G19} from the decision process. We also experienced problems with learning what decisions were made.

In sprint planning in the standard Scrum process, developers and PO negotiate to about what should be done in the sprint. When the Sprint backlog is defined, the developers can chose how to do it themselves. In \gls{G19}, the \Gls{POT} made all decisions about what to do and how to do it. They also micro managed the delegation of tasks, and developers had to ask them for permission before doing any work, so there would not be merge conflicts and blocking.

This gave a lot of overhead, especially in small tasks, and gave a lot of work to the \gls{POT}. The micro management could be avoided with better user stories and better overview of the tasks.

The micro management also gave some students a lot of power while others just had to do what they were told. It should not be the PO that decides how to do the work, as he is not a project leader.

The problems could be avoided if the PO and process definition roles was in meta groups instead og development teams. This would slow down the communication speed, but include all teams in the decisions.  S